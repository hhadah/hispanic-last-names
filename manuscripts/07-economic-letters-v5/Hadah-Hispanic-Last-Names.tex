%%
%% Copyright 2019-2021 Elsevier Ltd
%%
%% This file is part of the 'CAS Bundle'.
%% --------------------------------------
%%
%% It may be distributed under the conditions of the LaTeX Project Public
%% License, either version 1.2 of this license or (at your option) any
%% later version.  The latest version of this license is in
%%    http://www.latex-project.org/lppl.txt
%% and version 1.2 or later is part of all distributions of LaTeX
%% version 1999/12/01 or later.
%%
%% The list of all files belonging to the 'CAS Bundle' is
%% given in the file `manifest.txt'.
%%
%% Template article for cas-sc documentclass for
%% single column output.

\documentclass[a4paper,fleqn]{cas-sc}
\usepackage{setspace} % Load the setspace package for line spacing

% Set line spacing to double spaced
\doublespacing

% \usepackage{mathptmx} % To set the font to Times New Roman
% \let\cdot\relax
\usepackage{amssymb}

\usepackage{graphicx}
\usepackage{caption}

% Define a custom command to set the font for all tables
\newcommand{\tablefont}{\fontfamily{ptm}\selectfont}

% If the frontmatter runs over more than one page
% use the longmktitle option.

%\documentclass[a4paper,fleqn,longmktitle]{cas-sc}

%\usepackage[numbers]{natbib}
\usepackage[authoryear]{natbib}
%\usepackage[authoryear,longnamesfirst]{natbib}

%%%Author macros
\def\tsc#1{\csdef{#1}{\textsc{\lowercase{#1}}\xspace}}
\tsc{WGM}
\tsc{QE}
%%%

% Uncomment and use as if needed
%\newtheorem{theorem}{Theorem}
%\newtheorem{lemma}[theorem]{Lemma}
%\newdefinition{rmk}{Remark}
%\newproof{pf}{Proof}
%\newproof{pot}{Proof of Theorem \ref{thm}}

\usepackage{lscape}
\usepackage{threeparttablex}
\usepackage{longtable}
% Allow line breaks with \\ in specialcells
\newcommand{\specialcell}[2][c]{%
    \begin{tabular}[#1]{@{}c@{}}#2\end{tabular}
}

\usepackage{dcolumn}    % aligning decimals
    \newcolumntype{d}[1]{D{.}{.}{#1}}

\usepackage{siunitx}
%%%%%%%%%%% TiKz %%%%%%%%%%%%%%%%%%%%%
\usepackage{tikz}
\usetikzlibrary{shapes.geometric, arrows}

\tikzstyle{startstop} = [rectangle, rounded corners, minimum width=3cm, minimum height=1cm,text centered, draw=black, fill=red!30]

\tikzstyle{io} = [trapezium, trapezium left angle=70, trapezium right angle=110, minimum width=3cm, minimum height=1cm, text centered, text width =5cm, draw=black, fill=blue!30]

\tikzstyle{process} = [rectangle, minimum width=3cm, minimum height=1cm, text centered, text width =5cm, draw=black, fill=orange!30]

\tikzstyle{decision} = [diamond, minimum width=3cm, minimum height=1cm, text centered, draw=black, fill=green!30]

\tikzstyle{arrow} = [thick,->,>=stealth]
\begin{document}

\let\WriteBookmarks\relax
\def\floatpagepagefraction{1}
\def\textpagefraction{.001}

% Short title
\shorttitle{The Impact of Hispanic Last Names and Identity on Labor Market Outcomes}

% Short author
\shortauthors{Hadah}

% Main title of the paper
\title [mode = title]{The Impact of Hispanic Last Names and Identity on Labor Market Outcomes}

\author[]{Hussain Hadah}[orcid=https://orcid.org/0000-0002-8705-6386]
\ead{hhadah@tulane.edu}

\affiliation[]{organization={Department of Economics, Tulane University},
            addressline={6823 St. Charles Ave.},
            city={New Orleans},
            % citysep={}, % Uncomment if no comma needed between city and postcode
            % postcode={70118},
            state={LA 70118},
            country={United States}}




\begin{abstract}
\singlespacing % Set single spacing for the abstract
Do individuals with Hispanic names face labor market discrimination? This study analyzes the impact of Hispanic-sounding surnames on wages among inter-ethnic children with one White and one Hispanic parent. I find that individuals with Hispanic surnames receive less education, more likely to be unemployed, and often earn less, with a notable wage gap favoring those with White surnames. People born to Hispanic fathers and White mothers receive 0.3 years of education less than those born to White fathers and Hispanic mothers. Males born to Hispanic fathers and White mothers are 1 percentage point more likely to be unemployed, and they also earn 5 percentage points less than those born to White fathers and Hispanic mothers. The earnings gap is largely due to educational differences.
\end{abstract}

\begin{keywords}
\singlespacing % Set single spacing for keywords

 Economics of Minorities, Race, and Immigrants\sep Discrimination and Prejudice\\
 \JEL{J15, J64, J71}
\end{keywords}


\maketitle

%%%%%%%%%%%%%%%%%%%%%%%%%%%%%%%%%%%
% Main Text
%%%%%%%%%%%%%%%%%%%%%%%%%%%%%%%%%%%

\section{Introduction}

A substantial earnings gap exists between native-born White Hispanic men and White men, with Hispanic men earning 21\% less. While a significant portion of this gap is attributed to educational differences, discrimination may also play a role, potentially leading to reduced job opportunities, lower wages, and hindered assimilation \citep{duncan2006hispanics, duncan2018identifying, duncan2018socioeconomic}. This paper examines the impact of Hispanic last names on labor market and educational outcomes, contributing to the literature by isolating the effect of ethnicity-signaling surnames from other potentially confounding factors. By focusing on children of inter-ethnic marriages, this study provides a novel approach to understanding the mechanisms behind Hispanic-White earnings disparities and the potential role of name-based discrimination in the labor market.

This study investigates whether having a Hispanic last name affects educational and labor market outcomes. Previous research has identified gaps in education, health, and labor market outcomes between Hispanics and Whites \citep{antman2020ethnic, davilaChangesRelativeEarnings2008}. However, comparing Hispanic Whites to non-Hispanic Whites might artificially inflate the earnings gap. This is due to observable differences in education level, work experience, and immigration status.

Identifying discrimination is challenging due to unobservable factors affecting labor market outcomes. While audit studies provide insights, they primarily measure callback rates rather than full labor market outcomes \citep{bertrand2004emily}. This study compares children from inter-ethnic marriages: those of Hispanic fathers and White mothers (HW) to those of White fathers and Hispanic mothers (WH) \citep{rubinstein2014pride}. This approach leverages marriage selection based on observable characteristics \citep{averettBetterWorseRelationship2008}, allowing for the investigation of how ethnic signals, such as Hispanic last names, affect earnings.

I find that individuals with Hispanic-sounding last names receive 0.2 years less education and are more likely to be unemployed. The gap in education is equivalent to a 1.9\% reduction, and the unemployment gap is equal to a 14\% increase. The earnings gap between individuals with Hispanic-sounding last names and those with White last names is 5 percentage points. However, when controlling for education, the earnings gap reduces to 1 percentage point, suggesting that the earnings gap is largely due to differences in education.

\section{Data}\label{sec:data}

I use two datasets: the IPUMS Current Population Survey (CPS) Annual Social and Economic (ASEC) \citep{cps2019} and the 1960 to 2000 US censuses \citep{acs2019}. The CPS data (1994-2019) is used to study the effect of Hispanic-sounding last name on labor market outcomes. The census data is employed to construct synthetic parents using the method developed by \citet{rubinstein2014pride}.

The CPS sample is restricted to White, U.S.-born citizens aged 25-40, born between 1960 and 2000. Parents are classified as Hispanic if born in a Spanish-speaking country or Puerto Rico, and White if they are US-born. Four parental types are identified: (1) White father and White mother (WW), (2) White father and Hispanic mother (WH), (3) Hispanic father and White mother (HW), and (4) Hispanic father and Hispanic mother (HH). Table \ref{tab:combined-mat} shows the distribution of these types. WW children comprise 96\% of the sample, HH 3\%, and inter-ethnic children (WH and HW) 1.35\%. Summary statistics for each group are presented in Table \ref{tab:combined-sum}. Using the 1960-2000 censuses, I constructed a dataset of synthetic parents, including married White men and women born between 1920-1975. Hispanic individuals are defined as White and born in a Spanish-speaking country, while Whites are native-born. Table \ref{tab:combined-mat} shows the distribution of couple types: WW (96\%), HH (2\%), WH (<1\%), and HW (1\%). Summary statistics for the parents are presented in Table \ref{tab:combined-sum}.

\section{Empirical Strategy}\label{sec:emp_model}

In this section, I present my empirical strategy. The empirical strategy estimates the effect of having a Hispanic last name on educational outcomes and earnings. 

The difference in means between Hispanics and non-Hispanic Whites could result from either discrimination or differences in innate abilities, skills, and parental investments. It can also be caused by differences in innate abilities, skills, and parental investments. While controlling for observable skill measures, I compare children of inter-ethnic marriages, HW and WH. WH and HW children are more similar in characteristics but provide employers, and the labor market, with different signals.\footnote{WH and HW children are both half White, half Hispanic.} WH children will have a non-Hispanic last name, while HW children will have a Hispanic last name. This is a method developed by \citet{rubinstein2014pride}.

\subsection{Estimating the effect of having a Hispanic last name}

In this section I restrict the sample to WH and HW groups. Let $Y_{ist}$ be the outcome of interest for person $i$ in state $s$ at time $t$. $HW_{ist}$ is an indicator variable for the type of parents person $i$ has. $X_{ist}$ is a vector of controls that includes age and numbers of hours worked, $\gamma_{st}$ are state-year fixed effects (FE), and $\phi_{ist}$ represents the error term. The equation for this strategy is written as follows:
\begin{equation} \label{eq:1a}
Y_{ist} = \beta_{1} HW_{ist} + X_{ist} \pi + \gamma_{st} + \phi_{ist}
\end{equation}

$\beta_{1}$ is the coefficient of interest in this specification. $\beta_{1}$ represents the gaps in outcomes between children of inter-ethnic marriages who have a Spanish-sounding last name versus a White last name. 

\subsection{Threats to Identification}

The central assumption of my estimation strategy is that individuals born to HW parents have comparable characteristics to their WH peers, particularly in areas crucial to the labor market. This assumption faces two main challenges. First, Selection in the marriage market: Given the potential societal penalties associated with Hispanic identity, the characteristics of White women who marry Hispanic men may be unique. Second, Differential parental influence: Fathers and mothers may impact human capital accumulation differently \citep{kimball2009risk,magruder2010intergenerational}.

These concerns are mitigated by assortative matching in marriages. A substantial body of literature indicates strong selection in marriages based on traits \citep{averettBetterWorseRelationship2008, beckerTreatiseFamily1993,duncanIntermarriageIntergenerationalTransmission2011}. This suggests that inter-ethnic WH and HW marriages likely consist of partners with similar characteristics, supporting the comparability of their children in terms of labor market outcomes.

A potential measurement error arises from using parents' birthplace in the CPS data as a proxy for both ethnicity and last name. However, this concern is minimal as most Hispanics from 1960 to 2000 were first-generation immigrants. Census data shows only 3\% of native-born Americans identified as Hispanic, making it highly improbable for an inter-ethnic child to have a native-born, second-generation+ Hispanic father.\footnote{Self-reported Hispanic identity among first-generation Hispanic immigrants and native-borns are based on the author’s calculations from the 1960-2000 Census. The sample includes Whites, who are married, and are between the ages 25 and 40.} Empirical evidence from the CPS and US Census will be used to evaluate these concerns and support the validity of the identification strategy.

\section{From the Data: The Differences Between HW and WH Couples}\label{sec:hw-wh-couples-data}

In this section, I examine the data to validate my empirical strategy, focusing on educational and economic outcomes for four ethnic groupings—White White (WW), White Hispanic (WH), Hispanic White (HW), and Hispanic Hispanic (HH)—as shown in Table \ref{tab:combined-sum}. The data indicates marriage selection, with smaller differences between HW and WH children compared to the other groups. This similarity makes HW and WH children a more suitable comparison for analyzing labor market discrimination against Hispanics. Using the synthetic parents sample constructed from Census data, we observe that WW couples have the highest household education levels (24.95 years), while HH couples have the lowest (17.13 years). Inter-ethnic couples fall in between, with WH households totaling 22.68 years of schooling and HW households slightly behind at 21.50 years. Notably, HW wives are more educated than their WH peers, which could have implications for the educational attainment of HW children, as mothers play a crucial role in human capital accumulation \citep{kimball2009risk,magruder2010intergenerational}.

Regarding labor market outcomes, WW households have the highest log total family income (10.75), followed by WH (10.65), HW (10.60), and HH (10.42) households. The earnings gap between HW and WH men is marginal, with HW men earning 4\% less. However, HW women slightly out-earn WH women, with log hourly earnings of 1.75 and 1.73, respectively. This reversal in typical gender earning patterns suggests that HW women contribute more to their families' economic stability, potentially benefiting their children's development and future labor market prospects. Despite HW mothers' higher education and income levels, HW children complete, on average, 0.4 fewer years of education than their WH peers (Table \ref{tab:combined-sum}), potentially pointing to discrimination or barriers in educational access for HW children.

\section{Results}\label{sec:results}

My analysis of White, U.S.-born Hispanics ages 25-40 (Table \ref{tab:lastname-ed-reg}) reveals that HW children receive 0.2 years less education than WH children---that represents a 1.5\% reduction. While there is no significant difference in high school dropout rates, HW children are 2 percentage points less likely to earn an associate degree and 3 percentage points less likely to earn a bachelor's degree, which is equal to a 15\% reduction. These gaps are larger for HW women in higher education. These disparities suggest potential barriers or discrimination in access to higher education for HW children, particularly women.

I show in Table \ref{tab:lastnamereg} a 1 percentage point employment gap between HW and WH workers, and a 5 percentage point crude earnings gap. However, when controlling for education, the earnings gap reduces to a statistically insignificant 1 percentage point. I also find that HW children are 1 percentage point more likely to be unemployed than WH children. The gap in unemployment is equivalent to a 14\% increase and is statistically significant even after controlling for education. These results suggest that having a Hispanic last name negatively impacts labor market outcomes, particularly in terms of employment.

\section{Conclusion}\label{sec:con1}

This study examines discrimination against Hispanics in the U.S. labor market, focusing on the impact of Hispanic-sounding last name and identification on education and earnings. This research finds that inter-ethnic individuals with Spanish-sounding last names receive 0.2 years less education and are more likely to be unemployed. HW children earn 5 percentage points less than WH children, but this gap becomes statistically insignificant when controlling for education.

These results align with existing literature on disparities in educational access \citep{bergman2018education,gaddis2024racial}. While the earnings gap diminishes when controlling for education, this doesn't necessarily indicate an absence of discrimination, as education itself can be influenced by bias. Moreover, educational differences cannot explain the gaps in unemployment, suggesting that discrimination may play a role in labor market outcomes of individuals with Hispanic-sounding last name. Further research is needed to comprehensively understand the earnings gaps between Hispanics and Whites, particularly considering that mothers of HW children have higher education and earnings, which theoretically should lead to better outcomes for their children.


\section*{Declaration of Competing Interest}
The author declares that he has no relevant or material financial interests that relate to the research described in this paper.

\section*{Data availability}

Data is available \href{https://github.com/hhadah/hispanic-last-names/tree/main/data/datasets}{here}.

\newpage
\clearpage

\bibliographystyle{cas-model2-names}
\bibliography{bibliography}
\newpage
\clearpage
%table1
\begin{table}[H]
\centering
\caption{Number of Children and Couples by Parental and Couples' Type}
\label{tab:combined}
\resizebox{\linewidth}{!}{
\begin{threeparttable}
\begin{tabular}{lcccc}
\toprule
\multicolumn{1}{c}{ } & \multicolumn{4}{c}{Parental/Couples' Type} \\
\cmidrule(l{3pt}r{3pt}){2-5}
 & \specialcell{White Father/ \\ White Mother \\ White Husband/ \\ White Wife \\ (WW)} 
 & \specialcell{White Father/ \\ Hispanic Mother \\ White Husband/ \\ Hispanic Wife \\ (WH)} 
 & \specialcell{Hispanic Father/ \\ White Mother \\ Hispanic Husband/ \\ White Wife \\ (HW)} 
 & \specialcell{Hispanic Father/ \\ Hispanic Mother \\ Hispanic Husband/ \\ Hispanic Wife \\ (HH)}\\
\midrule
\textbf{\specialcell{Observations \\ (Children)}} 
& \specialcell{6,421,328\\0.96} & \specialcell{39,048\\0.01} & \specialcell{51,277\\0.01} & \specialcell{179,827\\0.03} \\
\midrule
\textbf{\specialcell{Observations \\ (Couples)}} 
& \specialcell{5,141,737\\0.96} & \specialcell{33,097\\0.01} & \specialcell{37,847\\0.01} & \specialcell{119,749\\0.02} \\
\bottomrule
\end{tabular}
\begin{tablenotes}
\item[1] Source (Children): Current Population Surveys (CPS) 1994-2019.
\item[2] Source (Couples): 1960-2000 Census.
\item[3] The sample includes Whites, who are married and between the ages 25 and 40. Ethnicity of a person's parents are identified by the parent's place of birth. A parent is Hispanic if she/he was born in a Spanish-speaking country. A parent is White if she/he was born in the United States.
\end{tablenotes}
\end{threeparttable}}
\end{table}


\newpage
\clearpage

%table3
\begin{landscape}
\begin{ThreePartTable}
\begin{TableNotes}[flushleft]
\small
\item[1] Source: 1950-2000 Census (synthetic parents), 1994-2019 CPS (children's outcomes)
\item[2] Synthetic parents: native-born US citizens, White, ages 25-40, with kids
\item[3] Children: native-born US citizens, White, ages 25-40, 1994-2019. WH, HW, HH restricted to self-identified Hispanic/Latino
\item[4] Parent ethnicity by birthplace: Hispanic if Spanish-speaking country, White if US
\item[5] Mean (SD) or Difference (SE). *** p<0.01, ** p<0.05, * p<0.1
\end{TableNotes}
\begin{longtable}{@{}l*{6}{c}@{}}
\caption{Summary Statistics by Couple Type \label{tab:combined}}\\
\toprule
& \multicolumn{4}{c}{Father's and Mother's Ethnicities} & \multicolumn{2}{c}{Differences} \\
\cmidrule(lr){2-5} \cmidrule(l){6-7}
Variables & WW & WH & HW & HH & HH - WW & HW - WH \\
\midrule
\endfirsthead
\caption[]{Combined Summary Statistics by Couple Type \textit{(continued)}}\\
\toprule
Variables & WW & WH & HW & HH & HH - WW & HW - WH \\
\midrule
\endhead
\midrule
\multicolumn{7}{r@{}}{\textit{(Continued on Next Page...)}}\\
\endfoot
\bottomrule
\insertTableNotes
\endlastfoot
\multicolumn{7}{@{}l}{\textbf{Panel A: Synthetic Parents}}\\
Husband's education & 12.58 & 11.82 & 10.33 & 8.91 & -3.67*** & -1.49** \\
 & (2.88) & (3.75) & (4.40) & (4.25) & (0.01) & (0.02) \\
Wife's education & 12.36 & 10.71 & 11.01 & 8.68 & -3.68*** & 0.29** \\
 & (2.40) & (3.97) & (3.44) & (4.00) & (0.01) & (0.02) \\
Total HH education & 24.95 & 22.68 & 21.50 & 17.69 & -7.26** & -1.18** \\
 & (4.77) & (6.91) & (6.99) & (7.40) & (0.01) & (0.04) \\
Log Total Family Income & 10.75 & 10.65 & 10.60 & 10.42 & -0.33*** & -0.05*** \\
 & (0.57) & (0.67) & (0.68) & (0.66) & (0.00) & (0.01) \\
Husband's Log Hourly Earnings & 1.74 & 1.76 & 1.72 & 1.55 & -0.19*** & -0.04*** \\
 & (0.83) & (0.87) & (0.88) & (0.80) & (0.00) & (0.01) \\
Wife's Log Hourly Earnings & 1.60 & 1.73 & 1.75 & 1.51 & -0.09*** & 0.02** \\
 & (0.93) & (0.88) & (0.89) & (0.79) & (0.01) & (0.02) \\
Number of Children & 3.84 & 4.05 & 4.28 & 4.29 & 0.44*** & 0.23** \\
 & (1.44) & (1.70) & (1.84) & (1.72) & (0.00) & (0.01) \\
\addlinespace
\multicolumn{7}{@{}l}{\textbf{Panel B: Children's Education}}\\
Men's education & 13.82 & 13.4 & 13.07 & 12.87 & -0.94*** & -0.32** \\
 & (2.42) & (2.38) & (2.27) & (2.3) & (0.01) & (0.03) \\
Women's education & 14.06 & 13.62 & 13.26 & 13.22 & -0.84*** & -0.36** \\
 & (2.37) & (2.39) & (2.35) & (2.39) & (0.01) & (0.02) \\
Men's HS Dropout Rate & 0.35 & 0.45 & 0.44 & 0.44 & 0.09*** & -0.01*** \\
 & (0.48) & (0.5) & (0.5) & (0.5) & (0) & (0.01) \\
Women's HS Dropout Rate & 0.36 & 0.45 & 0.46 & 0.46 & 0.1*** & 0.01*** \\
 & (0.48) & (0.5) & (0.5) & (0.5) & (0) & (0.01) \\
Men's Associate Rate & 0.15 & 0.15 & 0.12 & 0.11 & -0.03*** & -0.03*** \\
 & (0.36) & (0.36) & (0.32) & (0.32) & (0) & (0) \\
Women's Associate Rate & 0.19 & 0.17 & 0.14 & 0.15 & -0.04*** & -0.03*** \\
 & (0.39) & (0.37) & (0.35) & (0.36) & (0) & (0) \\
Men's BA Rate & 0.32 & 0.23 & 0.19 & 0.17 & -0.15*** & -0.04*** \\
 & (0.47) & (0.42) & (0.39) & (0.38) & (0) & (0.01) \\
Women's BA Rate & 0.36 & 0.28 & 0.22 & 0.22 & -0.14*** & -0.06*** \\
 & (0.48) & (0.45) & (0.41) & (0.41) & (0) & (0) \\
Men's Prof. Degree Rate & 0.09 & 0.07 & 0.05 & 0.04 & -0.05*** & -0.02*** \\
 & (0.28) & (0.25) & (0.21) & (0.19) & (0) & (0) \\
Women's Prof. Degree Rate & 0.1 & 0.08 & 0.07 & 0.06 & -0.04*** & -0.01*** \\
 & (0.3) & (0.26) & (0.25) & (0.23) & (0) & (0) \\
\addlinespace
\multicolumn{7}{@{}l}{\textbf{Panel C: Children's Employment and Earnings}}\\
Men's Unemployment Rate & 0.04 & 0.06 & 0.07 & 0.07 & 0.02*** & 0.01*** \\
 & (0.8) & (0.76) & (0.74) & (0.75) & (0.00) & (0.00) \\
Women's Unemployment Rate & 0.04 & 0.06 & 0.07 & 0.06 & 0.02*** & 0.01*** \\
 & (0.81) & (0.24) & (0.75) & (0.76) & (0.00) & (0.00) \\
Men's Log Hourly Earnings & 2.51 & 2.43 & 2.42 & 2.42 & -0.09*** & -0.00** \\
 & (0.45) & (0.46) & (0.44) & (0.43) & (0) & (0.02) \\
Women's Log Hourly Earnings & 2.32 & 2.31 & 2.28 & 2.31 & -0.02*** & -0.03** \\
 & (0.49) & (0.45) & (0.45) & (0.42) & (0) & (0.02) \\
Men's Log Annual Earnings & 10.29 & 10.09 & 10.06 & 10.01 & -0.28*** & -0.03** \\
 & (1.01) & (1.05) & (0.97) & (1.04) & (0.01) & (0.03) \\
Women's Log Annual Earnings & 10.13 & 10.04 & 10.02 & 10.01 & -0.12*** & -0.02** \\
 & (0.78) & (0.79) & (0.72) & (0.73) & (0.01) & (0.03) \\
\end{longtable}
\end{ThreePartTable}
\end{landscape}

\newpage
\clearpage

\begin{table}[H]
\centering\centering
\caption{Effect of Having Hispanic Last Name on Educational Outcomes\label{tab:lastname-ed-reg}}
\centering
\begin{threeparttable}
\begin{tabular}[t]{lcccc}
\toprule
  & \specialcell{(1) \\ Years of \\ Education} & \specialcell{(2) \\ High School \\ Dropout} & \specialcell{(3) \\ Associate Degree} & \specialcell{(4) \\ Bachelor Degree}\\
\midrule
\addlinespace[0.5em]
\multicolumn{5}{l}{\textit{Panel A: Full Sample}}\\
\midrule \hspace{1em}$HW_{ist}$ & -0.20*** & 0.01 & -0.02*** & -0.03***\\
\hspace{1em} & (0.05) & (0.01) & (0.01) & (0.01)\\
\hspace{1em}Observations & 88377 & 90027 & 66927 & 90027\\
\addlinespace[0.5em]
\multicolumn{5}{l}{\textit{Panel B: Women}}\\
\midrule \hspace{1em}$HW_{ist}$ & -0.25*** & 0.02 & -0.03*** & -0.04***\\
\hspace{1em} & (0.06) & (0.01) & (0.01) & (0.01)\\
\hspace{1em}Observations & 46516 & 47302 & 34334 & 47302\\
\addlinespace[0.5em]
\multicolumn{5}{l}{\textit{Panel C: Men}}\\
\midrule \hspace{1em}$HW_{ist}$ & -0.16** & 0.00 & -0.02** & -0.02\\
\hspace{1em} & (0.07) & (0.01) & (0.01) & (0.01)\\
\hspace{1em}Observations & 41861 & 42725 & 32593 & 42725\\
Full Sample's Mean & 13.48 & 0.43 & 0.15 & 0.26\\
Women's Mean & 13.58 & 0.44 & 0.16 & 0.27\\
Men's Mean & 13.38 & 0.43 & 0.14 & 0.24\\
\bottomrule
\multicolumn{5}{l}{\rule{0pt}{1em}* p $<$ 0.1, ** p $<$ 0.05, *** p $<$ 0.01}\\
\end{tabular}
\begin{tablenotes}
\item[1] {\setstretch{1.0}\footnotesize{This table includes the estimation results of equation (\ref{eq:1a}). All regressions include state-year fixed effects.}}
\item[2] {\setstretch{1.0}\footnotesize{HW is an indicator variable that is equal to 1 if a person is the child of a Hispanic-father and White-mother.}}
\item[3] {\setstretch{1.0}\footnotesize{Standard errors are clustered on the state level.}}
\end{tablenotes}
\end{threeparttable}
\end{table}


\newpage
\clearpage

\begin{table}[H]
\centering
\caption{Effect of Having Hispanic Last Name on Employment and Log Annual Earnings}
\label{tab:lastnamereg}
\begin{threeparttable}
\begin{tabular}{lcccc}
\toprule
  & \specialcell{(3) \\ Log Annual \\ Earnings} & \specialcell{(4) \\ Log Annual \\ Earnings} & \specialcell{(2) \\ Unemployment} & \specialcell{(3) \\ Unemployment} \\
\midrule
$HW_{ist}$ & \num{-0.05}* & \num{-0.02} & \num{0.01}** & \num{0.01}* \\
 & (\num{0.02}) & (\num{0.02}) & (\num{0.00}) & (\num{0.00}) \\
Constant &  &  &  & \\
 &  &  &  & \\
\midrule
\textit{Controlling for:} & & & & \\
Education &  & X &  & X \\
Observations & \num{3621} & \num{3621} & \num{38090} & \num{38090} \\
\bottomrule
\multicolumn{5}{l}{\rule{0pt}{1em}* p $<$ 0.1, ** p $<$ 0.05, *** p $<$ 0.01}\\
\end{tabular}
\begin{tablenotes}
\item[1] {\setstretch{1.0}\footnotesize{This table includes the estimation results of equation (\ref{eq:1a}).}}
\item[2] {\setstretch{1.0}\footnotesize{HW is an indicator variable that is equal to 1 if a person is the child of a Hispanic-father and White-mother.}}
\item[3] {\setstretch{1.0}\footnotesize{The sample is restricted to prime-age men for unemployment and full-time full-year wage and salary workers for earnings.}}
\item[4] {\setstretch{1.0}\footnotesize{I control for hours worked, age, and state-year fixed effects for earnings. I control for age, and state-year fixed effects for earnings.}}

\item[5] {\setstretch{1.0}\footnotesize{Standard errors are clustered at the state level.}}
\end{tablenotes}
\end{threeparttable}
\end{table}


\end{document}

