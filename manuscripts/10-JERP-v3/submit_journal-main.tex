\documentclass[12pt,english]{article}

%%%%%%%%%%%%%%%%%%%%%%%%%
%% SEC: PACKAGEs MAIN  %%
%%%%%%%%%%%%%%%%%%%%%%%%%
\usepackage{mathpazo}
\usepackage{eulervm}
\usepackage{amsmath}
\usepackage{amssymb}
\usepackage[utf8]{inputenc}
\usepackage[T1]{fontenc}
\usepackage{color}
% \usepackage[monochrome]{xcolor}
\usepackage{babel}
\usepackage{csquotes}
\usepackage{setspace}
\usepackage{graphicx}
\usepackage{booktabs,dcolumn}
\usepackage{array}
\usepackage{multirow}
\usepackage[font=singlespacing, skip=3pt]{caption}
\usepackage[usenames,dvipsnames,svgnames,table]{xcolor}
%\usepackage[usenames,dvipsnames,svgnames,table,monochrome]{xcolor} % test black and white color results, 2018-11-21 08:48
\usepackage{float}
\usepackage[export]{adjustbox}[2011/08/13]
\usepackage{enumitem}
% \usepackage{subfig}
%\usepackage[nolists, tablesfirst, nomarkers]{endfloat}
% \usepackage[unicode=true,pdfusetitle,
%  bookmarks=true,bookmarksnumbered=false,bookmarksopen=false,
%  breaklinks=false,pdfborder={0 0 1},backref=false,colorlinks=false]
%  {hyperref}
\usepackage[colorlinks=true, linkcolor=blue, citecolor=blue, plainpages=false, pdfpagelabels=true, urlcolor=blue]{hyperref}
\usepackage{geometry}
\usepackage{ragged2e}
\usepackage{appendix}
\usepackage{subcaption} % For subfigures

\usepackage{makecell}
\usepackage{adjustbox}
\usepackage{siunitx}
\usepackage{indentfirst}
\newcolumntype{d}{S[input-symbols = ()]}
\usepackage{lscape}
\usepackage{longtable}
\newcommand{\source}[1]{\caption*{Source: {#1}} }
\usepackage{threeparttable} % For tables with notes
\usepackage{threeparttablex} % For extending threeparttable functionality

% Redefine the footnoterule command
\usepackage[hang,flushmargin]{footmisc}
\renewcommand{\footnoterule}{%
    \kern-3pt % Adjust the vertical position of the line
    \hrule width 0.2\columnwidth height 0.4pt % Width and height of the line
    \kern2.6pt % Adjust the spacing between the line and the footnotes
}

\renewcommand{\thetable}{\arabic{table}}
\renewcommand{\thefigure}{\arabic{figure}}

% *****************************************************************
% Estout LaTeX wrapper
% *****************************************************************

%%Original code developed by Jörg Weber: see
%% https://www.jwe.cc/2012/03/stata-latex-tables-estout/
%% andwho
%% https://www.jwe.cc/blog/


\let\estinput=\input % define a new input command so that we can still flatten the document

\newcommand{\estwide}[3]{
		\vspace{.75ex}{
			%\textsymbols% Note the added command here
			\begin{tabular*}
			{\textwidth}{@{\hskip\tabcolsep\extracolsep\fill}l*{#2}{#3}}
			\toprule
			\estinput{#1}
			\bottomrule
			\addlinespace[.75ex]
			\end{tabular*}
			}
		}	

\newcommand{\estauto}[3]{
		\vspace{.75ex}{
			%\textsymbols% Note the added command here
			\begin{tabular}{l*{#2}{#3}}
			\toprule
			\estinput{#1}
			\bottomrule
			\addlinespace[.75ex]
			\end{tabular}
			}
		}

% Allow line breaks with \ in specialcells
\newcommand{\specialcell}[2][c]{%
    \begin{tabular}[#1]{@{}c@{}}#2\end{tabular}
}

% \newcommand{\sym}[1]{\rlap{#1}}% Thanks David Carlisle


%%%%%%%%%%% End of wrapper %%%%%%%%%%%%%%%%%%%%%
    
%%%%%%%%%%% TiKz %%%%%%%%%%%%%%%%%%%%%
\usepackage{tikz}
\usetikzlibrary{shapes.geometric, arrows}

\tikzstyle{startstop} = [rectangle, rounded corners, minimum width=3cm, minimum height=1cm,text centered, draw=black, fill=red!30]

\tikzstyle{io} = [trapezium, trapezium left angle=70, trapezium right angle=110, minimum width=3cm, minimum height=1cm, text centered, text width =5cm, draw=black, fill=blue!30]

\tikzstyle{process} = [rectangle, minimum width=3cm, minimum height=1cm, text centered, text width =5cm, draw=black, fill=orange!30]

\tikzstyle{decision} = [diamond, minimum width=3cm, minimum height=1cm, text centered, draw=black, fill=green!30]

\tikzstyle{arrow} = [thick,->,>=stealth]

% Citation comes loading other packages, so that footnotes follow biblatex-chicago style
%%%%%%%%%%%%%%%%%%%%%%%%
%% SEC: Bibliography %%
%%%%%%%%%%%%%%%%%%%%%%%%
\usepackage[authordate,
backend=biber,
doi=only,
isbn=false,
sorting=nyt,
maxcitenames=3,
minbibnames=7,
maxbibnames=7,
uniquename=false,
sortcites=true]{biblatex-chicago}

\bibliography{references.bib}

\AtEveryBibitem{\clearlist{note}\clearlist{language}\clearlist{issn}} % clears issn
\AtEveryBibitem{%
	\ifentrytype{online}{%
		\clearfield{urlyear}
		\clearfield{urlmonth}
		\clearfield{urlday}
		\clearfield{note}
		\clearlist{language}
	}{%
		\clearfield{eprint}%
% 		\clearfield{url}
		\clearfield{urlyear}
		\clearfield{urlmonth}
		\clearfield{urlday}
		\clearfield{note}
		\clearlist{language}
	}
}%



% Responses to Referees
\usepackage{blindtext}
\newcommand{\aptxspc}{1.8}
\newcommand{\rrquote}{1.1}
\newcommand{\rrxspc}{1.5}

\begin{document}

% SUBMISSION COVER
\thispagestyle{empty}
\begingroup
  \doublespacing
  \centering
  \LARGE REVISE AND RESUBMIT\\[0.25em]
  \LARGE JOURNAL OF ECONOMICS, RACE, AND POLICY \\[0.25em]
  \LARGE SUBMISSION ID: 33481294-df2a-4cce-80ee-c20f2b9975ac \\[1.0em]
\endgroup
This document and the associated contents in the revise and resubmit version of manuscript 33481294-df2a-4cce-80ee-c20f2b9975ac for the Journal of Economics, Race, and Policy. I include:
\begin{enumerate}
    \item Title page information
    \begin{itemize}
        \item Title and Author
        \item Abstract
        \item Keywords
        \item Acknowledgment
    \end{itemize}
    \item The manuscript
    \begin{itemize}
        \item Main manuscript text
        \item Main manuscript references
        \item Main manuscript figures
        \item Main manuscript tables
    \end{itemize}  
    \item \hyperref[r&r:responses]{Responses to the editor and referees comments}
    \item The online appendix, which is separately paged. It includes an online appendix title page, appendix A, appendix B, appendix C, and references for citations in the appendix.
\end{enumerate}
\clearpage

%%%%%%%%%%%%%%%%%%%%%%%%%%%%%%%%%%%%%%%%
% Part I. Title Page, Authors, Abstract, etc. 
%%%%%%%%%%%%%%%%%%%%%%%%%%%%%%%%%%%%%%%%
% Load in title page information, title, author information, abstract, etc are stored here

%%%%%%%%%%%%%%%%%%%%%%%%%%%%%%%%%%%%%%%%
% 1. Define Keywords, JEL
%%%%%%%%%%%%%%%%%%%%%%%%%%%%%%%%%%%%%%%%
\newcommand{\PAPERKEYWORDS}{\textbf{Keywords}: Economics of Minorities, Race, and Immigrants; Discrimination and Prejudice}
\newcommand{\PAPERJEL}{\textbf{JEL}: J71; J64; J15}

%%%%%%%%%%%%%%%%%%%%%%%%%%%%%%%%%%%%%%%%
% 2. Define Title
%%%%%%%%%%%%%%%%%%%%%%%%%%%%%%%%%%%%%%%%
% \newcommand{\PAPERTITLE}{\href{https://hhadah.github.io/hispanic-last-names/my_paper/Hadah-last-names-draft.pdf}{The Impact of Hispanic Last Names and Identity on Labor Market Outcomes}}
\newcommand{\PAPERTITLE}{The Impact of Hispanic Last Names and Identity on Labor Market Outcomes}

%%%%%%%%%%%%%%%%%%%%%%%%%%%%%%%%%%%%%%%%
% 3. Define Authors contact information
%%%%%%%%%%%%%%%%%%%%%%%%%%%%%%%%%%%%%%%%
\newcommand{\AUTHORHADAH}{Hussain Hadah}
\newcommand{\AUTHORHADAHURL}{https://orcid.org/0000-0002-8705-6386}
\newcommand{\AUTHORHADAHINFO}{\href{\AUTHORHADAHURL}{\AUTHORHADAH}: Department of Economics, Tulane University, 6823 St. Charles Ave., New Orleans, LA 70118, United States (e-mail: \href{mailto:hhadah@tulane.edu}{hhadah@tulane.edu}, phone: +1-602-393-8077)}


%%%%%%%%%%%%%%%%%%%%%%%%%%%%%%%%%%%%%%%%
% 4. Define Thanks
%%%%%%%%%%%%%%%%%%%%%%%%%%%%%%%%%%%%%%%%
\newcommand{\ACKNOWLEDGMENTS}{
I thank Professors Willa Friedman, Chinhui Juhn, Vikram Maheshri, and Yona Rubinstein for their support and advice. I also thank Aimee Chin, Steven Craig, German Cubas, Elaine Liu, Fan HADAH, and the participants of the Applied Microeconomics Workshop at the University of Houston, and European Society for Population Economics (ESPE) for helpful feedback.}

%%%%%%%%%%%%%%%%%%%%%%%%%%%%%%%%%%%%%%%%
% 5. Define Abstract
%%%%%%%%%%%%%%%%%%%%%%%%%%%%%%%%%%%%%%%%
\newcommand{\PAPERABSTRACT}{
Do individuals with ethnically Hispanic names face labor market discrimination? In this study, I analyze the impact of Hispanic-sounding surnames on wages, considering inter-ethnic children and the distinct traits of native and Hispanic surnames. I examine the earnings of individuals with one White and one Hispanic parent, finding that those with Hispanic surnames often experience wage disparities. My findings reveal a notable wage gap favoring individuals with White surnames. Although males born to Hispanic fathers and White mothers earn 5 percentage points less than those born to White fathers and Hispanic mothers, this difference can be attributed to educational variances. However, this shouldn't imply an absence of discrimination; it may reflect discrimination in human capital accumulation. Furthermore, I investigate the impact of self-identifying as Hispanic on earnings. Men with a Spanish-sounding last name who identify as Hispanic earn notably less than those who do not, largely due to educational differences.
\PAPERJEL}

%%%%%%%%%%%%%%%%%%%%%%%%%%%%%%%%%%%%%%%%
% 6. Define citation or availability of latest draft
%%%%%%%%%%%%%%%%%%%%%%%%%%%%%%%%%%%%%%%%
\newcommand{\PAPERDOIURL}{https://doi.org/10.1086/711654}
\newcommand{\PAPERINFO}{

 \url{\PAPERDOIURL}.
}
\pagenumbering{roman}
\setcounter{page}{1}

% Part I.a.1 Title
\section*{TITLE AND AUTHORS}
\subsection*{TITLE}
\begin{itemize}[label={}, leftmargin=*]
    \item Title: \textbf{\PAPERTITLE}
\end{itemize}
% Part I.a.2 Author info
\subsection*{AUTHORS}
\begin{itemize}[label={}, leftmargin=*]
    \item List of authors:
    \begin{enumerate}
        \item \AUTHORHADAHINFO
    \end{enumerate}
    \item Corresponding author:
    \begin{itemize}
        \item \textbf{\AUTHORHADAHINFO}
    \end{itemize}
\end{itemize}
\clearpage 

% Part I.b Abstract
\doublespacing
\section*{ABSTRACT}
\PAPERABSTRACT
\clearpage 

% Part I.c Keywords
\doublespacing
\section*{KEYWORDS}
\PAPERKEYWORDS
\clearpage 

% statements

\doublespacing
\section*{Acknowledgement}
\ACKNOWLEDGMENTS
\clearpage 

\subsection*{Data Availability Statement}
The data that support the findings of this study will be openly available to all researchers after the review process. For immediate information on the data and / or computer programs used for this study, please contact Hussain Hadah at hhadah@tulane.edu.

\subsection*{Funding Statement}

The author has no funding for the research, authorship, and/or publication of this article to report.

\subsection*{Conflict of Interest Disclosure}
The author declares that they have no conflict of interest with respect to the publication of this manuscript.

\clearpage 


%%%%%%%%%%%%%%%%%%%%%%%%%%%%%%%%%%%%
% Part II. Manuscript main text
%%%%%%%%%%%%%%%%%%%%%%%%%%%%%%%%%%%%
\pagenumbering{arabic}
\setcounter{page}{1}
\renewcommand*{\thefootnote}{\arabic{footnote}}
\doublespacing
\begingroup
  \centering
  \Large Manuscript Text\\[1em]
\endgroup
% Manuscript main text
%%%%%%%%%%%%%%%%%%%%%%%%%%%%%%%%%%%
% Main Text
%%%%%%%%%%%%%%%%%%%%%%%%%%%%%%%%%%%

\section{Introduction}

A large literature documents substantial earnings gaps across race and ethnicity \autocite{bayer2018divergent, charles2008prejudice}. Hispanics constitute a large and growing portion of the population in the United States. As the number of Hispanics increases, determining whether ethnic discrimination affects their labor market outcomes becomes increasingly crucial \autocite{chettyUnitedStatesStill2014, chettyEffectsExposureBetter2016,chettyFadingAmericanDream2017}. Specifically, it is essential to investigate the extent to which a person's Hispanic ethnicity may influence their employment prospects, wages, and career advancement. Understanding these labor market outcomes is critical, as they directly impact broader societal issues such as assimilation and economic mobility. These factors serve as key indicators of how successfully Hispanics can integrate into society and ascend the socioeconomic ladder.

In this paper, I answer the following questions. Does having a Hispanic last name affect educational outcomes? Does having a Hispanic last name affect labor market outcomes? I aim to show that comparing Hispanic Whites to non-Hispanic Whites might create an artificially higher earnings gap since the two groups differ in many observable characteristics.\footnote{By identifying as Hispanic, I refer to individuals who self-report their ethnicity as Hispanic on surveys or other data collection instruments.} \footnote{Observable characteristics refer to factors that can be measured and quantified, such as education level, work experience, and immigration status.} Others have attempted to compare how native-born White Hispanics fare to non-Hispanic Whites and foreign-born Hispanics. In \textcite{antman2020ethnic,antmanEthnicAttritionObserved2016,antmanEthnicAttritionObserved2016a,antmanEthnicAttritionAssimilation2020}, the authors compare the health and educational outcomes of Hispanic Whites to non-Hispanic Whites and native-born Hispanics to foreign-born Hispanics. They find gaps in education and health between Hispanics and Whites. They also find that native-born Hispanics are more likely than their foreign-born counterparts to report poor health. \textcite{davilaChangesRelativeEarnings2008} documents many gaps in labor market outcomes between Hispanics and Whites. They attribute a big part of this gap to differences in education, experience, immigration status, and regional differences. 

This study builds upon and improves the traditional Oaxaca-Blinder-Kitagawa decomposition approach often used in labor market discrimination studies \autocite{oaxaca1973male,blinder1973wage,kitagawa1955components}. While these decomposition methods have provided valuable insights into earnings gaps between different groups, including Hispanics and non-Hispanic Whites \autocite{davilaChangesRelativeEarnings2008}, they are limited in their ability to account for unobserved differences between the groups. My approach, by comparing children of inter-ethnic marriages, provides a more closely matched comparison group, allowing me to better isolate the effect of having a Hispanic last name on labor market outcomes.

This methodological improvement is particularly timely and relevant given the rapidly changing demographics of the United States. The US population is growing increasingly diverse, with significant implications for labor market dynamics and potential discrimination. The proportion of non-Whites has increased by more than 10 percentage points from 13 percent in 1995 to 23 percent in 2019.\footnote{The proportion of non-Whites and Hispanics is calculated using the Current Population Survey (CPS).} Native-born White Hispanic men earn 21\% less than White men, although a substantial portion of the earnings gap is due to educational differences between Hispanics and Whites \autocite{duncan2006hispanics, duncan2018identifying, duncan2018socioeconomic}. Some earnings differences may also be due to discrimination which will have negative consequences. For example, discrimination against Hispanics can lead to reduced job opportunities, lower wages, and hinder assimilation. In this paper, I examine the role of having a Hispanic last name and identifying as Hispanic on labor market outcomes. 

Using this novel approach, I find significant effects of having Hispanic last names on both educational and labor market outcomes. The results show that individuals with Hispanic last names face substantial disadvantages in educational attainment and earnings compared to their counterparts with non-Hispanic names. Specifically, individuals with Hispanic last names complete 0.2 years of education less than those with non-Hispanic names, even when controlling for family background. Individuals with Hispanic last names are also 1 percentage point more likely to be unemployed  and earn 5 percentage points less than those with non-Hispanic names. These findings suggest that ethnic discrimination continues to play a role in shaping economic opportunities for Hispanics in the United States, highlighting the need for targeted policies to address these persistent disparities.

Identifying discrimination is difficult because of factors that affect labor market outcomes that are unobservable to economists---such as unobserved skills and separating it from prejudice and stereotypes. One strategy used by researchers is audit or resume studies. \textcite{bertrand2004emily} conducted an audit study where identical resumes were sent to employers with White and Black-sounding names. This approach, however, has its drawbacks. Audit studies only observe callbacks, not wages. 

This study utilizes a method developed by \textcite{rubinstein2014pride}. I compare children from inter-ethnic marriages. More precisely, I compare children of Hispanic fathers and White mothers (henceforth HW) to children of White fathers and Hispanic mothers (henceforth  WH). This approach stems from the fact that there is a strong selection among many characteristics, and thus, marriages are not random. Couples match on several observable characteristics like income, schooling, socio-economic background, etc. \autocite{averettBetterWorseRelationship2008, averettEconomicRealityBeauty1996}.\footnote{For more on assortative mating see \autocite{beckerTheoryMarriagePart1973, beckerTheoryMarriagePart1974, beckerTreatiseFamily1993, browningCollectiveUnitaryModels2006, chiapporiFatterAttractionAnthropometric2012}} Children of HW and WH marriages have more similar observable characteristics than children of endogamous/homogamous marriages--- i.e., White fathers-White mothers and Hispanic fathers-Hispanic mothers. Moreover, children from a Hispanic father and White mother household will have a Hispanic last name from their fathers, enabling the investigation of how ethnic signals, such as having a Hispanic last name, affect annual log earnings.

The main identifying assumption of my empirical strategy depends on the assumption that people born to HW parents are similar to their WH peer in all observable---and unobservable---aspects and characteristics that are important in the labor market. Consequently, the only difference between the two groups is the variation in which group is more likely to have a Hispanic sounding last name. Children from mixed ethnic backgrounds may appear physically similar to those from single ethnicity backgrounds, but the influence of family dynamics and upbringing, crucial in developing skills and personal characteristics, varies with the pattern of mixed ethnic marriages. Particularly in a society where Hispanic ancestry could be perceived negatively, it raises the question of what kind of White women would choose Hispanic men. Furthermore, even if such unions were formed randomly, children from White-Hispanic homes might benefit from more favorable family conditions than those from Hispanic-White homes, considering Whites generally have stronger socio-economic backgrounds than Hispanics. These factors introduce doubts regarding whether children from Hispanic-White families receive comparable familial support and influences, either genetically or environmentally, as those from White-Hispanic families.

Previous studies have also used names as a proxy for race and ethnicity \autocite{fryer2004causes, rubinstein2014pride, bertrand2004emily}. \textcite{fryer2004causes} point out that names can be a predictor of a person's race. Specifically, they provide a rising pattern among Blacks having different names than Whites. They did, however, find that having a Black name, after controlling for the home environment at birth, does not affect their labor market outcomes. \textcite{rubinstein2014pride} compared the children of mixed marriages between Sephardic and Ashkenazi Jews in Israel.\footnote{Ashkenazi and Sephardic are two distinct Jewish ethnic groups.} They found that workers with Sephardic last names earn substantially less than those with Ashkenazi last names. \textcite{bertrand2004emily} conducted an audit study by sending employers identical resumes that differ in the ethnic and racial signal of a name (Black sounding name versus a White sounding one). They found that resumes with Black-sounding names received substantially fewer callbacks than their White counterparts. 

Moreover, audit studies in education economics investigated the effect of racial and ethnic signal on access to education. \textcite{bergman2018education} found that students with Hispanic sounding names received 2 percentage points fewer responses from schools than students with White sounding names. \textcite{janssen2022guidance} found that guidance counselors restricted Asian students from advanced opportunities. \textcite{gaddis2024racial} found significant discrimination in interactions with principals against Hispanic and Chinese American families. Finally, \textcite{bourabain2023school} found more evidence of discrimination in access to education against students with underprivileged backgrounds in the Flemish education system. This paper contributes to this literature by providing evidence that discrimination in access to education could lead to lower earnings for Hispanics. 

The rest of the paper is organized as follows. In Section \ref{sec:data}, I describe the data used in this paper. In Section \ref{sec:emp_model}, I present the empirical strategy. In Section \ref{sec:results}, I present the results from the estimation of the two specifications. Finally, in Section \ref{sec:con1}, I conclude.

\section{Data}\label{sec:data}

I use two datasets:  the Integrated Public Use Microdata Series (IPUMS) Current Population Survey (CPS) Annual Social and Economic (ASEC) \autocite{cps2019} and the 1960 to 2000 US censuses \autocite{acs2019}. 

I use the CPS data set to study the effect of having a Hispanic last name on a person's labor market outcomes. I take advantage of the fact that the CPS asks parents' place of birth, ethnicity, and race. The data spans the period between 1994--- the earliest sample to ask about a parent's place of birth--- to 2019. Moreover, since the CPS does not provide data on parents' characteristics, essential to determine the family background, I use the census to construct synthetic parents. The census offers a larger sample of potential parents. Similar to the CPS, the 1960 to 2000 censuses ask about the person's place of birth and the individual's race and ethnicity. I employ this information to construct "synthetic parents" using a method developed by \textcite{rubinstein2014pride}. I construct the synthetic parents by linking husbands and wives in the census data to each other. I assume that parents have children between the ages of 25 and 40. I then link these synthetic parents using the birth year of child, and parents' places of birth to the year of birth of the "children" and the parents' places of birth in the CPS sample.

\subsection{Children of the four parental types}

I use the CPS for my primary analysis of the effect of having a Hispanic last name on earnings. I restrict my sample to Whites, United States-born citizens aged 25 to 40 and born between 1960 to 2000. Taking advantage of data on parents' place of birth, I divide the sample into four groups depending on their parent's ethnicity. Mothers or fathers are Hispanic if they were born in a Spanish-speaking country and Puerto Rico, and White if they were born in the United States.\footnote{The list of Spanish Speaking countries include: Argentina, Bolivia, Chile, Colombia, Costa Rica, Cuba, Dominican Republic, Ecuador, El Salvador, Guatemala, Equatorial Guinea, Honduras, Mexico, Nicaragua, Panama, Paraguay, Peru, Spain, Uruguay, Venezuela.} Therefore, an observation can be the product of four types of parents: 
\begin{enumerate}
\item White father and White mother (hereafter WW) 
\item White father and Hispanic mother (hereafter WH)
\item Hispanic father and White mother (hereafter HW)
\item Hispanic father and Hispanic mother (hereafter HH).
\end{enumerate}

The distribution of the four types of children is presented in Table \ref{tab:mat1}. The majority of the sample (96\%) is WW children. The second biggest group is HH, which constitutes 3\% of the sample. Inter-ethnic children, WH and HW, make up 1.35\% of the sample with 90,325 observations. Even though WH and HW are only 1.35\% of the sample, I have plenty of observations to carry out an analysis. The summary statistics for the children of the four types of marriages are presented in Table \ref{tab:c&p2}. Children of WW marriages (Column 1) do better on every measure while children of HH parents (Column 4) do worse than other children on every measure. Children of WH (Column 2) and HW (Column 3) marriages fall in between WW and HH children.

 
\subsection{Synthetic parents}

Using the 1960 to 2000 censuses, I constructed a data set of synthetic parents. The sample includes married White men and women. Even though the census asks a person whether they are Hispanic or not, I took advantage of the questions on place of birth to create a proxy for ethnicity. I consider a Hispanic person as White and born in a Spanish-speaking country. Consequently, Whites in the sample are people who are White and native-born. Using the information provided in the census, I can link husbands and wives with each other. I assume that parents have children between the ages of 25 and 40. Therefore, my sample consists of married White men and women with children that were born in the 1920 to 1975 cohorts\footnote{The construction of "synthetic parents" follows the method used by \textcite{rubinstein2014pride}.}.

I show the distribution of the four types of couples in Table \ref{tab:mat2}. White husbands and White wives (WW) make up the majority of couples in the sample, 96\% (5,141,737 couples). Hispanic husbands and wives (HH) are the second largest group making up 2\% (119,749 couples) of all couples. White husbands and Hispanic wives (WH) couples are less 1\% (33,097 couples) of the sample and Hispanic husbands and White wives (HW) are also 1\% (37,847 couples). I present the summary statistics of the parents in Table \ref{tab:synth}.

\section{Empirical Approach}\label{sec:emp_model}

In this section, I present my empirical Approach. The specification estimates the effect of having a Hispanic last name on educational outcomes and earnings. The difference in means between Hispanics and non-Hispanic Whites could result from discrimination. It can also be caused by differences in innate abilities, skills, and parental investments. While controlling for observable skill measures, I compare children of inter-ethnic marriages, HW and WH. WH and HW children are more similar in characteristics but provide employers, and the labor market, with different signals.\footnote{WH and HW children are both half White, half Hispanic.} WH children will have a non-Hispanic last name, while HW children will have a Hispanic last name. This is a method developed by \textcite{rubinstein2014pride}.

This approach allows me to control for several key factors that could influence labor market outcomes. Importantly, I control for parental education and income using my synthetic parents data, which addresses concerns about intergenerational transmission of human capital. By comparing children from inter-ethnic marriages, I also implicitly control for many unobservable family characteristics that might affect labor market outcomes, as these families are likely to be more similar to each other than to families with two Hispanic or two non-Hispanic White parents.

\subsection{Estimating the Effect of Having a Hispanic Last Name}

In this section I restrict the sample to WH and HW groups. Let $Y_{ist}$ be the outcome of interest for person $i$ in state $s$ at time $t$. $HW_{ist}$ is an indicator variables for the type of parents of person $i$ has. $X_{ist}$ is a vector of controls that includes age and numbers of hours worked, $\gamma_{st}$ are state-year fixed effects (FE), and $\phi_{ist}$ represents the error term.\footnote{I also include parental characteristics of `synthetic' parents as controls in alternative specifications. These controls include income and education of `synthetic' parents.} The equation for this strategy is written as follows:
\begin{equation} \label{eq:1a}
Y_{ist} = \beta_{1} HW_{ist} + X_{ist} \pi + \gamma_{st} + \phi_{ist}
\end{equation}

$\beta_{1}$ is the coefficient of interest in this specification. $\beta_{1}$ represents the gaps in outcomes between children of inter-ethnic marriages who have a Spanish-sounding last name versus a White last name. If $\beta_{1} > 0$, then people with a Hispanic last name have better outcomes than people with a White last name. If $\beta_{1} < 0$, then people with a Hispanic last name have worse outcomes than people with a White last name.

Moreover, by comparing interethnic children, I can control for many unobservable family characteristics that might affect labor market outcomes. These families are likely to be more similar to each other than to families with two Hispanic or two non-Hispanic White parents. This approach allows me to isolate the effect of having a Hispanic last name on labor market outcomes, providing a more accurate estimate of the impact. This is an improvement over traditional Oaxaca-Blinder-Kitagawa decomposition methods, which are limited in their ability to account for unobserved differences between groups by providing a more suitable comparison group to measure discrimination \autocite{oaxaca1973male,blinder1973wage,kitagawa1955components}.

\subsection{Threats to Identification}

The central assumption underpinning my estimation strategy is based on the hypothesis that individuals born to a HW person exhibits comparable characteristics to their peers of WH descent, especially in areas vital to the labor market. This assumption includes similarities in educational background, skill sets, and work experiences that are significant determinants in employment opportunities, salary levels, and career advancement. 

Two primary reasons underscore that this assumption. First, there is a significant selection in the marriage market. Since belonging to the Hispanic out-group comes with a negative penalty in society, one might wonder who are the White women that would be willing to give their kids a Hispanic last-name that would potentially negatively influence their futures. Second, fathers and mothers influence human capital accumulation differently \autocite{kimball2009risk,magruder2010intergenerational}. If marriage was random, then WH households might be a better environment for children that HA. Using the CPS and the US Census, I will evaluate these empirical concerns. 

The previous two threats to identification could be addressed by the selection of partners among inter-ethnic couples. The average White person exhibits better observable characteristics than the Average Hispanic. Consequently, if marriages were random, the White husband in an WH marriage would have better traits than wife, and the White wife in the HW marriage would have the stronger traits. Random matching in marriages would pose a challenge to the identification strategy. This is because parents are not perfect substitutes in the inter-generational transmission of human capital. Consequently, differences in parental background could significantly affect children's labor market outcomes, potentially invalidating the comparison between HW and WH children.

Marriages, however, are not random. There is large body of empirical and theoretical work that indicates that marriages exhibits strong selection, or what is referred to as assortative matching, on traits \autocite{averettBetterWorseRelationship2008, averettEconomicRealityBeauty1996, beckerTheoryMarriagePart1973, beckerTheoryMarriagePart1974, beckerTreatiseFamily1993, browningCollectiveUnitaryModels2006, chiapporiFatterAttractionAnthropometric2012}. Also, \textcite{duncanIntermarriageIntergenerationalTransmission2011} shows a similar pattern of assortative matching among Mexicans in the United States. These papers suggest that inter-ethnic WH and HW marriages will, on average, consist of partners sharing similar characteristics. Consequently, the literature would predict that both WH and HW parents will be similar in all aspects that are relevant to a child's labor market outcomes. I will go over the empirical evidence from the data to show that this is actually the case.

Another threat to identification could arise from measurement error in using the place of birth of parents in the CPS data as a proxy for a parent's ethnicity and last name. The CPS only asks about the place of birth of parents and not their ethnic or racial identity. Therefore, it is possible that a native-born father could be a second-generation or later immigrant from a Spanish-speaking country. However, this scenario is highly improbable. Most Hispanics from 1960 to 2000 were first-generation immigrants, and the number of second-generation or later was very small. This is supported by data from the 1960 to 2000 censuses (see Table \ref{tab:mat3}). Only 3\% of native-born Americans identified as Hispanic, indicating the small number of second-generation or later Hispanic immigrants in the sample. Therefore, the probability that an inter-ethnic child with a native-born father that is also a second generation+ Hispanic immigrant is very small.

\section{From the Data: The Differences Between HW and WH Couples}\label{sec:hw-wh-couples-data}

In this section, I explore the empirical data to affirm the validity of my empirical strategy, as presented in Table \ref{tab:synth}, which details the educational and economic profiles of parents from four different ethnic groupings—White White (WW), White Hispanic (WH), Hispanic White (HW), and Hispanic Hispanic (HH). This table meticulously lays out the average outcomes and discrepancies for each group, highlighting the impact of inter-ethnic marriages on children's prospects. These results show that there is selection in marriage, and that the differences between children born to HW and children born to WH are less severe that those between the other two groups. Consequently, comparing WH and HW children to each other to analyze discrimination against Hispanics in the labor market.

Using the ``synthetic'' parents sample that I constructed using the Census data, I examine the family background of the different type of children. WW couples have higher education, 12.58 years for husbands and 12.36 for wives. As a household, WW couples have 24.95 years of schooling. Husbands in HH marriages have 8.64 years of education, while women have 8.49 years of schooling. As a household, HH couples have 17.13 years of education. As predicted, the inter-ethnic couples exhibit matching in which partners marry people similar to them. WH husbands have 11.82 years of education, while wives have an average of 10.71 years. WH household attained a total of 22.68 years of schooling in total. HW husbands have 10.33 years of education, while wives have an average of 11.01 years. HW household attained a total of 21.50 years of schooling in total. Both Hispanic men and women in inter-ethnic couples marry white spouses that are, on average, more educated the the average HH couple. More importantly for human capital accumulation, the mother in the HW marriages---the children that have a Spanish sounding last-name---are more educated than their WH peers. Since mothers could be more important in the a child's human capital accumulation and education, this would suggest that a child with a Hispanic last-name should complete more years of education \autocite{gould2020does}.

The data reveals that WW couples have the highest total household education, amounting to 24.95 years, significantly surpassing the 17.69 years of HH couples. This stark contrast underlines the substantial educational divide between these groups. Inter-ethnic couples, WH and HW, display intermediate educational achievements, with WH households totaling 22.68 years of education and HW households slightly behind at 21.50 years. Notably, HW husbands are less educated (10.33 years) compared to WH husbands (11.82 years), yet HW wives surpass their WH counterparts with 11.01 years of education, suggesting a balance in educational attainment within these marriages. This factor is particularly pertinent for children with Hispanic last names who might derive greater benefits from their mother's higher education.

In terms of labor market performance, WW households boast the highest log total family income at 10.75, while HH households fall at the lower end with 10.42. Among inter-ethnic couples, WH households have a slightly higher total income of 10.65 compared to HW households at 10.60. Notably, the difference in husbands' log hourly earnings between HW and WH is marginal, indicating HW men earn 4\% less than their WH counterparts. Furthermore, HW women actually surpass WH women in earnings, with HW wives earning 1.75 and WH wives earning 1.73 in log hourly earnings, respectively. This reversal in the typical earning pattern not only speaks to the closing economic disparities between these groups but also implies potentially greater economic contributions from HW women to their families, which could be advantageous for children and to the developments of traits that are important in the labor market.

The table also sheds light on fertility trends, noting that WW couples have less children than HH couples, reflecting broader socio-economic and cultural patterns. The difference in fertility between HW and WH couples is positive but significantly lower than the difference in fertility between HH and WW couples.

The evidence presented in this section supports a robust empirical strategy, revealing significant selection in marriage and distinctive educational and income patterns among different types of families. Inter-ethnic couples exhibit comparable levels of education and earnings. This suggests that the children of HW families are likely positioned for better educational outcomes, critical in understanding discrimination in the labor market. Notably, the disparities between HW and WH families are considerably less pronounced than those between other groups, emphasizing the importance of comparing these children directly. Such comparisons shed light on the nuanced dynamics of ethnicity, education, and economic outcomes in inter-ethnic marriages. 

% \footnote{I present in Tables \ref{tab:synthmex} and \ref{tab:syntnonmex} the summary statistics which detail the educational and economic profiles of parents from four different ethnic groupings—White White (WW), White Hispanic (WH), Hispanic White (HW), and Hispanic Hispanic (HH) on sub-samples of Hispanics of Mexican and non-Mexican ancestries. I find similar results that describe a selection into inter-ethnic marriages among the two groups.} 

Despite higher levels of education and income among Hispanic White (HW) mothers compared to White Hispanic (WH) mothers, suggesting an expectation of greater educational attainment for HW children, data reveals a discrepancy. Specifically, HW children complete, on average, 0.4 fewer years of education compared to their WH peers (refer to Table \ref{tab:c&p2}). This gap may suggest potential discrimination or barriers in educational access for HW children.

\section{Results}\label{sec:results}

In this section, I present the results from estimating the specification presented in equation \ref{eq:1a}. I estimate the mean educational outcomes of White, Native-born (results in Table \ref{tab:lastname-ed-reg}), Hispanics aged 25-40, and mean unemployment of White, Native-born, Hispanics men aged 25-40 (results in Table \ref{tab:lastnamereg-emp}). I estimate the mean wages of White, Native-born, Hispanic men aged 25-40 who are employed full-time and full-year (FTFY) as waged and salaried workers (results in Table \ref{tab:lastnamereg}).

First, I find that an inter-ethnic person with a Spanish-sounding last name receives fewer total years of education than an inter-ethnic person who has a White last name. A person with a Spanish-sounding last name receives 0.2 years of education less than a person with a White last name, have the same high school dropout rate as a person with a White last name, and 2 percentage points less likely to have an associate degree. People with a Spanish-sounding last name are 3 percentage points less likely to have a bachelor's degree.   

Second, I find that an inter-ethnic person with a Spanish-sounding last names are 1 percentage point more likely to be unemployed and earn less than an inter-ethnic who has a White last name. A person with a Spanish-sounding last name earns 5 percentage points less than a person with a White last name. In other words, by comparing inter-ethnic children, a person with a Hispanic last name earns 5 percentage points less than someone with a White last name. However, more than half of the earnings gaps could be explained by educational differences. When I control for education, the last name effect decreases to a statistically insignificant 1 percentage points earnings gap. I also find a significant earnings gap between those that identify as Hispanic. 

\subsection{The Effect of Having a Hispanic Last Name on Educational Outcomes}

I provide the results of the estimation of equation \ref{eq:1a} in Table \ref{tab:lastname-ed-reg}. I estimate the mean educational outcomes of White, U.S.-born Hispanics ages 25-40. I also restrict the sample to children of HW and WH parents. The omitted group is children of WH parents. Column 1 in Table \ref{tab:lastname-ed-reg} is the difference in total years of education between HW children and their WH peers. Column 2 is the difference in the probability of having been a high school dropout. Column 3 is the difference in the probability of having an associate degree. Column 4 is the difference in the probability of having a bachelor's degree. All regressions include controls for age, parental education and income, and state-year fixed effects.

Overall, there is a significant gap in total years of education between HW and WH children. HW children receive 0.2 years of education less than WH children. The gap between HW and WH women is larger than the gap between HW and WH men. Women with a Hispanic last name receive 0.25 years of education less than WH women. The gap between HW and WH men is equal to 0.16 years.

Even though there is a significant, albeit small, gap in total years of education between HW and WH children, there is no significant gap in the probability of being a high school dropout. The gap between HW and WH high school dropouts is equal to a statistically insignificant 1 percentage point. The same is true for HW women (2 percentage points) and men (0 percentage points).

However, there is a notable difference when it comes to higher education outcomes. HW children are 2 percentage points less likely to earn an associate degree and 3 percentage points less likely to earn a bachelor's degree compared to their WH peers. HW children are 2 percentage points less likely to earn an associate degree compared to their WH peers, representing a 13.3\% reduction relative to the WH associate degree rate of 15\%. HW children are 3 percentage points less likely to earn a bachelor's degree compared to their WH peers, which is a 13\% reduction relative to the WH bachelor's degree rate of 23\%. These gaps are slightly larger for HW women, who are 3 percentage points less likely to earn an associate degree and 4 percentage points less likely to earn a bachelor's degree. For HW men, the gap is 2 percentage points for an associate degree and statistically insignificant for a bachelor's degree.

These results suggest that while the overall educational attainment gap between HW and WH children is small in terms of years of education, the disparities become more pronounced when considering higher education milestones, particularly for HW women. However, given the fact that HWs have more educated mothers than WHs, we would expect them to have higher levels of educational attainment \autocite{kimball2009risk, gould2020does}. This could indicate potential barriers or discrimination in access to higher education for HW children, particularly.

\subsection{The Effect of Having a Hispanic Last Name on Labor Market Outcomes}

I provide the results to the estimation of equation \ref{eq:1a} in Tables \ref{tab:lastnamereg-emp} and \ref{tab:lastnamereg} on unemployment and log earnings. I estimate the mean unemployment of White, Native-born, Hispanics men aged 25-40. I estimate the mean wages of White, U.S.-born, Hispanic men aged 25-40 who are employed FTFY as waged and salaried workers. I also restrict the sample to children of HW and WH (omitted) parents. Column 1 in Table \ref{tab:lastnamereg} is the average crude earnings gap in log annual earnings between HW workers and their WH peers. In the next 4 columns, I introduce the results with controls for hours worked, state FE, year FE, age FE, Education FE, and parental background. 

In addition to examining earnings, I also analyzed the effect of having a Hispanic last name on unemployment rates. Table \ref{tab:lastnamereg-emp} presents the results of this analysis. Column 1 shows that individuals with Hispanic last names (HW) have a 1 percentage point higher unemployment rate compared to those with White last names. This gap persists even after controlling for age, state fixed effects, year fixed effects, and state-year fixed effects (Column 2), though the significance level drops. When education is included as a control (Column 3), the gap remains at 1 percentage point. Finally, after controlling for parental background (Column 4), the 1 percentage point difference in unemployment rates becomes statistically insignificant. These results suggest that while there is an initial unemployment gap associated with having a Hispanic last name, much of this difference can be explained by factors such as education and parental background. The mean unemployment rate for individuals with Hispanic last names (HW) is 7\% across all specifications.

Overall, the crude gap between HW and WH workers is equal to 5 percentage points (Table \ref{tab:lastnamereg} column 1). An inter-ethnic with a Hispanic last name earns 5 percentage points less than an inter-ethnic with a White last name. Even after controlling for hours worked, and including state, year, and age FEs in the estimation, the gap stays at 5 percentage points. This gap, however, could be entirely explained by educational differences. An inter-ethnic with a Hispanic last name earns 1 percentage points less than an inter-ethnic with a White last name, but the result is statistically insignificant. 

% Since Hispanics are a very heterogeneous group, I conduct a heterogeneity analysis on different samples of Hispanics. To increase the sample size of my analysis I estimate equation \ref{eq:1a} using weekly earnings as a dependent variable. Weekly earnings are available in all the monthly CPS surveys and not just the March Supplement. I present the results to these estimations in tables \ref{tab:lastnamereg-weekearm}-\ref{tab:lastnamereg-weekearm-cub}. 

% First, in Table \ref{tab:lastnamereg-weekearm}, I present the results of estimating equation \ref{eq:1a} using weekly earnings as the dependent variable for the full sample. Similar to the previous analysis, I find that a person with a Hispanic last name earns 4 percentage points less than a person without a Hispanic last name. This gap could be explained by educational differences. Second, in Table \ref{tab:lastnamereg-weekearm-mex}, I present the results for a sample of Mexican Hispanics. I find that a Mexican with a Hispanic-sounding last name earns 3 percentage points less than a Mexican with a native-sounding last name. This gap becomes an imprecise zero after controlling for education and parental background. Third, in Table \ref{tab:lastnamereg-weekearm-nonmex}, I present the results for a sample of non-Mexican Hispanics. The gap between non-Mexicans with Hispanic-sounding last names and those with native-sounding last names is similarly explained by educational differences. I find that a non-Mexican with a Hispanic-sounding last name earns 3 percentage points less than a non-Mexican with a native-sounding last name. This gap also becomes an imprecise zero after controlling for education and parental background. Finally, in Table \ref{tab:lastnamereg-weekearm-cub}, I present the results for a sample of Cubans. I find that a Cuban with a Hispanic-sounding last name earns 4 percentage points less than a Cuban with a native-sounding last name, but this gap is statistically insignificant.

\section{Conclusion}\label{sec:con1}

As the Hispanic population grows in the United States, studying discrimination against this group becomes increasingly important. In this paper, I examine discrimination against Hispanics in the labor market. More specifically, I examine the impact of Hispanic last names and Hispanic identification on annual log earnings. 

I compare the children of inter-ethnic marriages to study the labor and educational markets impact of having a Hispanic last name. I find that inter-ethnic people with Spanish-sounding last names receive 0.2 years of education less than inter-ethnic counterparts that have a native-sounding last name. When I compare the earnings of HW and WH children, which captures the effect of having a Hispanic last name, HW children are 1 percentage point more likely to be unemployed than WH children. HW children earn 5 percentage points less than WH children. Thus, by comparing inter-ethnic children, a person with a Hispanic last name makes 5 percentage points less than someone with a White last name. When I control for education, the last name effect decreases to a statistically insignificant 2 percentage points earnings gap. 

There is a large literature in education economics that shows disparities in access to education, which this paper contributes to. In an audit study of charter and traditional public schools, \textcite{bergman2018education,gaddis2024racial} find that students with Hispanic names---compared to students that are presumed White---are less likely to get a response to inquiries from schools. This could be an indication that people with Hispanic sounding names could end up lower access to education, including high value-added schools. Therefore, the earnings gap between people with Hispanic sounding last names and those with native sounding last names could be due to differences in access to education that are in turn due to discrimination. This paper provides further evidence, using observational data, that discrimination in access to education could lead to lower earnings for Hispanics. 

While the earnings gap between children of inter-ethnic parents with and without a Hispanic last name disappears when controlling for education, it does not necessarily indicate the absence of discrimination. Education itself can be influenced by bias, potentially resulting in divergent outcomes \autocite{bergman2018education,gaddis2024racial}. This is especially the case when parental characteristics indicate that people with a Hispanic-sounding last-names should in theory complete more years of education---since mothers of inter-ethnic children with a Hispanic last-names have more years of education and earn more. Consequently, further research is needed to comprehensively understand the earnings gaps between Hispanics and Whites.

%%%%%%%%%%%%%%%%%%%%%%%%%%%%%%%%%%%
% Part III. Bibliography
%%%%%%%%%%%%%%%%%%%%%%%%%%%%%%%%%%%
\newpage
\begingroup
\setstretch{1.0}
\setlength\bibitemsep{5.0pt}
\printbibliography[title=References for Manuscript]
\endgroup
\pagebreak

%%%%%%%%%%%%%%%%%%%%%%%%%%%%%%%%%%%%
% Part IV. Tables and figures
%%%%%%%%%%%%%%%%%%%%%%%%%%%%%%%%%%%%
%%%%%%%%%%%%%%%%%%%%%%%%%%%%%%%%%%%%
% Figures
%%%%%%%%%%%%%%%%%%%%%%%%%%%%%%%%%%%%

% Figure 1

\pagebreak
\newpage


\clearpage

%%%%%%%%%%%%%%%%%%%%%%%%%%%%%%%%%%%%
% Tables
%%%%%%%%%%%%%%%%%%%%%%%%%%%%%%%%%%%%
% Table 1
\begin{table}[t]
\tablefont
\caption{Number of Children by Parental Type \label{tab:mat1}}
%\resizebox{\linewidth}{!}{
\begin{threeparttable}
\begin{tabular}[t]{>{}lcccc}
\toprule
\multicolumn{1}{c}{ } & \multicolumn{4}{c}{Perental Type} \\
\cmidrule(l{3pt}r{3pt}){2-5}
  & \specialcell{White Father \\ White Mother} & \specialcell{White Father \\ Hispanic Mother} & \specialcell{Hispanic Father \\ White Mother} & \specialcell{Hispanic Father \\ Hispanic Mother}\\
\midrule
\textbf{\specialcell{Observations\\Share}} & \specialcell{6,421,328\\0.96} & \specialcell{39,048\\0.01} & \specialcell{51,277\\0.01} & \specialcell{179,827\\0.03}\\
\bottomrule
\end{tabular}
\begin{tablenotes}
\item[1] Source: Current Population Surveys (CPS) 1994-2019
\item[2] The sample includes Whites, who are married, and are between the ages 25 and 40. Ethnicity of a person's parents are identified by the parent's place of birth. A parent is Hispanic if she/he was born in a Spanish-speaking country. A parent is White if she/he was born in the United States.
\end{tablenotes}
\end{threeparttable}
%}
\end{table}


\newpage

\begin{table}[!h]

\caption{Summary statistics of outcomes using parent's place of birth only for those that self-identify as Hispanic \label{tab:c&p2}}
\centering
\resizebox{\linewidth}{!}{
\begin{threeparttable}
\begin{tabular}[t]{lcccccc}
\toprule
\multicolumn{1}{c}{ } & \multicolumn{4}{c}{Father's and Mother's Ethnicities} & \multicolumn{2}{c}{Differences} \\
\cmidrule(l{3pt}r{3pt}){2-5} \cmidrule(l{3pt}r{3pt}){6-7}
Variables & \specialcell{White Father \\ White Mother \\ (WW) \\ (i)} & \specialcell{White Father \\ Hispanic Mother \\ (WH) \\ (ii)} & \specialcell{Hispanic Father \\ White Mother \\ (HW) \\ (iii)} & \specialcell{Hispanic Father \\ Hispanic Mother \\ (HH) \\ (iv)} & \specialcell{HH - WW \\ (v)} & \specialcell{HW - WH \\ (vi)}\\
\midrule
\textbf{Panel A: Parent’s} & \textbf{} & \textbf{} & \textbf{} & \textbf{} & \textbf{} & \textbf{}\\
\hspace{1em}Husband’seducation (Total Years) & \specialcell{13.05\\(2.44)} & \specialcell{12.32\\(3.33)} & \specialcell{10.65\\(4.39)} & \specialcell{8.93\\(4.41)} & \specialcell{-4.11\\(0.02)} & \specialcell{-1.67\\(0.04)}\\
\hspace{1em}Wife’seducation (Total Years) & \specialcell{12.74\\(2.12)} & \specialcell{11.03\\(3.92)} & \specialcell{11.54\\(3.12)} & \specialcell{8.6\\(4.13)} & \specialcell{-4.13\\(0.02)} & \specialcell{0.51\\(0.04)}\\
\hspace{1em}Total Household seducation (Total Years) & \specialcell{25.78\\(4.08)} & \specialcell{23.35\\(6.51)} & \specialcell{22.19\\(6.69)} & \specialcell{17.54\\(7.83)} & \specialcell{-8.25\\(0.03)} & \specialcell{-1.16\\(0.07)}\\
\textbf{Panel B: Education} & \textbf{} & \textbf{} & \textbf{} & \textbf{} & \textbf{} & \textbf{}\\
\addlinespace
\hspace{1em}Men’s education (Total Years) & \specialcell{12.97\\(2.15)} & \specialcell{13.45\\(2.37)} & \specialcell{13.13\\(2.27)} & \specialcell{12.89\\(2.25)} & \specialcell{-0.08\\(0.01)} & \specialcell{-0.32\\(0.03)}\\
\hspace{1em}Women’s education (Total Years) & \specialcell{13.23\\(2.25)} & \specialcell{13.75\\(2.41)} & \specialcell{13.32\\(2.34)} & \specialcell{13.26\\(2.37)} & \specialcell{0.03\\(0.01)} & \specialcell{-0.43\\(0.03)}\\
\textbf{Panel C: Employment and Earnings} & \textbf{} & \textbf{} & \textbf{} & \textbf{} & \textbf{} & \textbf{}\\
\hspace{1em}Men’s Employment Rate & \specialcell{0.93\\(0.26)} & \specialcell{0.94\\(0.23)} & \specialcell{0.92\\(0.27)} & \specialcell{0.93\\(0.26)} & \specialcell{0.00\\(0.00)} & \specialcell{-0.02\\(0.00)}\\
\hspace{1em}Women’s Employment Rate & \specialcell{0.94\\(0.25)} & \specialcell{0.94\\(0.23)} & \specialcell{0.93\\(0.26)} & \specialcell{0.94\\(0.24)} & \specialcell{0.00\\(0.00)} & \specialcell{-0.02\\(0.00)}\\
\addlinespace
\hspace{1em}Men’s Log Hourly Earnings & \specialcell{2.4\\(0.44)} & \specialcell{2.41\\(0.45)} & \specialcell{2.4\\(0.44)} & \specialcell{2.41\\(0.43)} & \specialcell{0.01\\(0.01)} & \specialcell{-0.01\\(0.02)}\\
\hspace{1em}Women’s Hourly Earnings & \specialcell{2.26\\(0.43)} & \specialcell{2.32\\(0.45)} & \specialcell{2.27\\(0.45)} & \specialcell{2.3\\(0.41)} & \specialcell{0.04\\(0.01)} & \specialcell{-0.05\\(0.02)}\\
\hspace{1em}Men’s Log Annual Earnings & \specialcell{10.02\\(1.02)} & \specialcell{10.06\\(1.06)} & \specialcell{10.03\\(0.99)} & \specialcell{10\\(1.04)} & \specialcell{-0.02\\(0.01)} & \specialcell{-0.03\\(0.04)}\\
\hspace{1em}Women’s Hourly Earnings & \specialcell{9.44\\(1.59)} & \specialcell{9.55\\(1.59)} & \specialcell{9.47\\(1.55)} & \specialcell{9.52\\(1.52)} & \specialcell{0.08\\(0.02)} & \specialcell{-0.08\\(0.05)}\\
\bottomrule
\end{tabular}
\begin{tablenotes}
\item[1] The data is restricted to native-born United States citizens between 1994 and 2019 who are also White and between the ages of 25 and 40. I identify the ethnicity of a person's parents through the parent's place of birth. A parent is Hispanic if they were born in a Spanish-speaking country. A parent is White if they were born in the United States.
\item[2] In each column, I present the average statistics of the different types of people based on the ethnicities of their parents. In column one, I show the summary statistics of children of White fathers and White mothers. In column two, I present the summary statistics of children of White fathers and Hispanic mothers. In column three, I show the summary statistics of children of Hispanic fathers and White mothers. In column four, I present the summary statistics of children of Hispanic fathers and mothers.
\item[3] Columns five and six have data on the HH--WW gaps (column five) and the HW--WH gaps (column six).
\end{tablenotes}
\end{threeparttable}}
\end{table}


\newpage

\begin{table}[!h]

\caption{Couples' Type \label{tab:mat2}}
\centering
\resizebox{\linewidth}{!}{
\begin{threeparttable}
\begin{tabular}[t]{>{}lcccc}
\toprule
\multicolumn{1}{c}{ } & \multicolumn{4}{c}{Couples' Type} \\
\cmidrule(l{3pt}r{3pt}){2-5}
  & \specialcell{White Husband \\ White Wife} & \specialcell{White Husband \\ Hispanic Wife} & \specialcell{Hispanic Husband \\ White Wife} & \specialcell{Hispanic Husband \\ Hispanic Wife}\\
\midrule
\textbf{Observations} & \specialcell{1,286,731\\(0.97)} & \specialcell{7,178\\(0.01)} & \specialcell{7,606\\(0.01)} & \specialcell{20,911\\(0.02)}\\
\bottomrule
\end{tabular}
\begin{tablenotes}
\item[1] The data is restricted to people interviewed in 1970 and 1960 and also White and married. I identify the ethnicity of a person through their place of birth. A parent is Hispanic if they were born in a Spanish-speaking country. A parent is White if they were born in the United States.
\item[2] The table includes information on the proportion of the four types of synthetic parents that I have constructed.
\end{tablenotes}
\end{threeparttable}}
\end{table}


\newpage
\newgeometry{bottom=1.0in}

\begin{table}[H]

\caption{Summary statistics of synthetic parents by couple type \label{tab:synth}}
\centering
\resizebox{\linewidth}{!}{
\begin{threeparttable}
\begin{tabular}[t]{>{\raggedright\arraybackslash}p{5cm}cccccc}
\toprule
\multicolumn{1}{c}{ } & \multicolumn{4}{c}{Father's and Mother's Ethnicities} & \multicolumn{2}{c}{Differences} \\
\cmidrule(l{3pt}r{3pt}){2-5} \cmidrule(l{3pt}r{3pt}){6-7}
Variables & \specialcell{White \\ White \\ (WW) \\ (1)} & \specialcell{White \\ Hispanic \\ (WH) \\ (2)} & \specialcell{Hispanic \\ White \\ (HW) \\ (3)} & \specialcell{Hispanic \\ Hispanic \\ (HH) \\ (4)} & \specialcell{HH - WW \\ (5)} & \specialcell{HW - WH \\ (6)}\\
\midrule
Husband's education (Total Years) & \specialcell{12.75\\(0.61)} & \specialcell{11.77\\(1.87)} & \specialcell{10.25\\(2.44)} & \specialcell{8.64\\(2.01)} & \specialcell{-4.11***\\(0.00)} & \specialcell{-1.52***\\(0.00)}\\
Wife's education (Total Years) & \specialcell{12.47\\(0.56)} & \specialcell{10.40\\(2.23)} & \specialcell{11.11\\(1.75)} & \specialcell{8.49\\(1.92)} & \specialcell{-3.98***\\(0.00)} & \specialcell{0.70***\\(0.00)}\\
Total Household seducation (Total Years) & \specialcell{25.22\\(1.17)} & \specialcell{22.17\\(4.05)} & \specialcell{21.36\\(4.10)} & \specialcell{17.13\\(3.88)} & \specialcell{-8.09***\\(0.00)} & \specialcell{-0.81***\\(0.01)}\\
Log Total Family Income & \specialcell{10.72\\(0.09)} & \specialcell{10.55\\(0.26)} & \specialcell{10.46\\(0.27)} & \specialcell{10.27\\(0.21)} & \specialcell{-0.45***\\(0.00)} & \specialcell{-0.10***\\(0.00)}\\
Fertility & \specialcell{3.77\\(0.40)} & \specialcell{3.98\\(0.71)} & \specialcell{4.15\\(0.79)} & \specialcell{4.23\\(0.64)} & \specialcell{0.46***\\(0.00)} & \specialcell{0.17***\\(0.00)}\\
\bottomrule
\end{tabular}
\begin{tablenotes}
\item[1] Source: The 1950-2000 Census for synthetic parents, and 1994-2019 Current Population Surveys (CPS) for children's outcomes
\item[2] The data is restricted to native-born United States citizens who are also White, between the ages of 25 and 40, and have kids. I identify the ethnicity of a person's parents through the parent's place of birth. A parent is Hispanic if they were born in a Spanish-speaking country. A parent is White if they were born in the United States.
\end{tablenotes}
\end{threeparttable}}
\end{table}


\newpage

\begin{table}[H]
\centering\centering
\caption{Self-reported Hispanic Identity Among First-Generation Hispanic Immigrants and Native-Born \label{tab:mat3}}
\centering
\resizebox{\ifdim\width>\linewidth\linewidth\else\width\fi}{!}{
\begin{threeparttable}
\begin{tabular}[t]{>{}lcccc}
\toprule
  & \specialcell{Native Born Husband} & \specialcell{Spanish-Speaking \\ Place of Birth \\ Husband} & \specialcell{Native Born Wife} & \specialcell{Spanish-Speaking \\ Place of Birth \\ Wife}\\
\midrule
\textbf{\specialcell{Proportion White\\Proportion Hispanic}} & \specialcell{0.97\\0.03} & \specialcell{0.03\\0.97} & \specialcell{0.97\\0.03} & \specialcell{0.03\\0.97}\\
\bottomrule
\end{tabular}
\begin{tablenotes}
\item[1] Source: 1960-2000 Census
\item[2] The sample includes Whites, who are married, and are between the ages 25 and 40.
\end{tablenotes}
\end{threeparttable}}
\end{table}


\newpage

\begin{table}[H]
\centering\centering
\caption{Effect of Having Hispanic Last Name on Educational Outcomes\label{tab:lastname-ed-reg}}
\centering
\begin{threeparttable}
\begin{tabular}[t]{lcccc}
\toprule
  & \specialcell{(1) \\ Years of \\ Education} & \specialcell{(2) \\ High School \\ Dropout} & \specialcell{(3) \\ Associate Degree} & \specialcell{(4) \\ Bachelor Degree}\\
\midrule
\addlinespace[0.5em]
\multicolumn{5}{l}{\textit{Panel A: Full Sample}}\\
\midrule \hspace{1em}$HW_{ist}$ & -0.20*** & 0.01 & -0.02*** & -0.03***\\
\hspace{1em} & (0.05) & (0.01) & (0.01) & (0.01)\\
\hspace{1em}Observations & 88377 & 90027 & 66927 & 90027\\
\addlinespace[0.5em]
\multicolumn{5}{l}{\textit{Panel B: Women}}\\
\midrule \hspace{1em}$HW_{ist}$ & -0.25*** & 0.02 & -0.03*** & -0.04***\\
\hspace{1em} & (0.06) & (0.01) & (0.01) & (0.01)\\
\hspace{1em}Observations & 46516 & 47302 & 34334 & 47302\\
\addlinespace[0.5em]
\multicolumn{5}{l}{\textit{Panel C: Men}}\\
\midrule \hspace{1em}$HW_{ist}$ & -0.16** & 0.00 & -0.02** & -0.02\\
\hspace{1em} & (0.07) & (0.01) & (0.01) & (0.01)\\
\hspace{1em}Observations & 41861 & 42725 & 32593 & 42725\\
Full Sample's Mean & 13.48 & 0.43 & 0.15 & 0.26\\
Women's Mean & 13.58 & 0.44 & 0.16 & 0.27\\
Men's Mean & 13.38 & 0.43 & 0.14 & 0.24\\
\bottomrule
\multicolumn{5}{l}{\rule{0pt}{1em}* p $<$ 0.1, ** p $<$ 0.05, *** p $<$ 0.01}\\
\end{tabular}
\begin{tablenotes}
\item[1] {\setstretch{1.0}\footnotesize{This table includes the estimation results of equation (\ref{eq:1a}). All regressions include state-year fixed effects.}}
\item[2] {\setstretch{1.0}\footnotesize{HW is an indicator variable that is equal to 1 if a person is the child of a Hispanic-father and White-mother.}}
\item[3] {\setstretch{1.0}\footnotesize{Standard errors are clustered on the state level.}}
\end{tablenotes}
\end{threeparttable}
\end{table}


\newpage

\begin{table}[H]
\centering\centering
\caption{Effect of Having Hispanic Last Name on Employment \label{tab:lastnamereg-emp}}
\centering
\begin{threeparttable}
\begin{tabular}[t]{lcccc}
\toprule
  & \specialcell{(1) \\ Unemployment} & \specialcell{(2) \\ Unemployment} & \specialcell{(3) \\  Unemployment} & \specialcell{(4) \\  Unemployment}\\
\midrule
$HW_{ist}$ & \num{0.01}*** & \num{0.01}** & \num{0.01}* & \num{0.01}\\
 & (\num{0.00}) & (\num{0.00}) & (\num{0.00}) & (\num{0.00})\\
Constant & \num{0.05}*** &  &  & \\
 & (\num{0.00}) &  &  & \\
\midrule
\textit{Controlling for:} &  &  &  & \\
Age &  & X & X & X\\
State FE &  & X & X & X\\
Year FE &  & X & X & X\\
State-Year FE &  & X & X & X\\
Education &  &  & X & X\\
Parental Background &  &  &  & X\\
\specialcell{HW's Mean \\ Unemployment} & 0.07 & 0.07 & 0.07 & 0.07\\
Observations & \num{38090} & \num{38090} & \num{38090} & \num{38090}\\
\bottomrule
\multicolumn{5}{l}{\rule{0pt}{1em}* p $<$ 0.1, ** p $<$ 0.05, *** p $<$ 0.01}\\
\end{tabular}
\begin{tablenotes}
\item[1] {\setstretch{1.0}\footnotesize{This table includes the estimation results of equation (\ref{eq:1a}).}}
\item[2] {\setstretch{1.0}\footnotesize{HW is an indicator variable that is equal to 1 if a person is the child of a Hispanic-father and White-mother.}}
\item[3] {\setstretch{1.0}\footnotesize{The sample is restricted to prime-age men.}}
\item[4] {\setstretch{1.0}\footnotesize{Standard errors are clustered on the state level.}}
\end{tablenotes}
\end{threeparttable}
\end{table}


\newpage

\begin{table}[!h]

\caption{Effect of Having Hispanic Last Name \label{tab:lastnamereg}}
\centering
\resizebox{\linewidth}{!}{
\begin{threeparttable}
\begin{tabular}[t]{lcc}
\toprule
  & \specialcell{(1) \\ Log annual earnings} & \specialcell{(2) \\  Log annual earnings}\\
\midrule
$WH_{i}$ & \num{-0.14}*** & \num{-0.09}***\\
 & (\num{0.03}) & (\num{0.02})\\
$HW_{i}$ & \num{-0.20}*** & \num{-0.11}***\\
 & (\num{0.02}) & \vphantom{1} (\num{0.01})\\
$HH_{i}$ & \num{-0.24}*** & \num{-0.12}***\\
 & (\num{0.02}) & (\num{0.01})\\
\midrule
$HW_{i} - WH_{i}$ & -0.06** & -0.02\\
 & (0.03) & (0.02)\\
\midrule
Observations & \num{129359} & \num{129359}\\
State FE & X & X\\
Year FE & X & X\\
\textit{Controlling for:} &  & \\
Hours Worked & X & X\\
Age & X & X\\
Education &  & X\\
\bottomrule
\multicolumn{3}{l}{\rule{0pt}{1em}* p $<$ 0.1, ** p $<$ 0.05, *** p $<$ 0.01}\\
\end{tabular}
\begin{tablenotes}
\item[1] \footnotesize{This table includes the estimation results of equation (\ref{eq:1a}).}
\item[2] \footnotesize{The four groups stand for White Husband White Wife (WW), White Husband Hispanic Wife (WH), Hispanic Husband White (HW), and Hispanic Husband Hispanic Wife (HH).}
\item[3] \footnotesize{The sample is restricted to men working full-time full-year and are waged and salaried workers.}
\item[4] \footnotesize{Column one has the regression results when controlling for hours worked, age, and fixed effects. Column two has the results after controlling for education.}
\item[5] \footnotesize{Standard errors are clustered on the state level.}
\end{tablenotes}
\end{threeparttable}}
\end{table}


\newpage

\begin{table}[H]
\centering\centering
\caption{Effect of Having Hispanic Last Name (Log Weekly Earnings) \label{tab:lastnamereg-weekearm}}
\centering
\begin{threeparttable}
\begin{tabular}[t]{lccc}
\toprule
  & \specialcell{(1) \\ Log weekly \\ earnings} & \specialcell{(2) \\ Log weekly \\ earnings} & \specialcell{(3) \\  Log weekly \\ earnings}\\
\midrule
$HW_{ist}$ & \num{-0.04}*** & \num{-0.06}*** & \num{-0.03}**\\
 & (\num{0.01}) & (\num{0.01}) & (\num{0.01})\\
Constant & \num{6.65}*** &  & \\
 & (\num{0.03}) &  & \\
\midrule
\textit{Controlling for:} &  &  & \\
State FE &  & X & X\\
Year FE &  & X & X\\
State-Year FE &  & X & X\\
Age &  & X & X\\
Education &  &  & X\\
Observations & \num{7444} & \num{7444} & \num{7444}\\
\bottomrule
\multicolumn{4}{l}{\rule{0pt}{1em}* p $<$ 0.1, ** p $<$ 0.05, *** p $<$ 0.01}\\
\end{tabular}
\begin{tablenotes}
\item[1] {\setstretch{1.0}\footnotesize{This table includes the estimation results of equation (\ref{eq:1a}) where the dependent variable is log weekly earnings.}}
\item[2] {\setstretch{1.0}\footnotesize{HW is an indicator variable that is equal to 1 if a person is the child of a Hispanic-father and White-mother.}}
\item[3] {\setstretch{1.0}\footnotesize{The sample is restricted to men working full-time and are wage and salary workers.}}
\item[4] {\setstretch{1.0}\footnotesize{Column one has the regression results when controlling for hours worked, age, education, year and state fixed effects. Column two has the results after controlling for education.}}
\item[5] {\setstretch{1.0}\footnotesize{Standard errors are clustered on the state level.}}
\end{tablenotes}
\end{threeparttable}
\end{table}


% \newpage

% \begin{table}[!h]

\caption{Effect of Having Hispanic Last Name \label{tab:identreg}}
\centering
\resizebox{\linewidth}{!}{
\begin{threeparttable}
\begin{tabular}[t]{lcc}
\toprule
  & \specialcell{(1) \\ Log annual earnings} & \specialcell{(2) \\  Log annual earnings}\\
\midrule
$WH_{i} \times Hispanic_{i}$ & \num{-0.13}** & \num{-0.06}\\
 & (\num{0.05}) & \vphantom{2} (\num{0.05})\\
$HW_{i} \times Hispanic_{i}$ & \num{-0.16}*** & \num{-0.11}**\\
 & (\num{0.05}) & \vphantom{1} (\num{0.05})\\
$HH_{i} \times Hispanic_{i}$ & \num{-0.11}** & \num{-0.06}\\
 & (\num{0.05}) & \vphantom{2} (\num{0.04})\\
$WH_{i}$ & \num{0.03} & \num{0.01}\\
 & (\num{0.05}) & \vphantom{1} (\num{0.04})\\
$HW_{i}$ & \num{0.01} & \num{0.04}\\
 & (\num{0.05}) & (\num{0.05})\\
$HH_{i}$ & \num{-0.06} & \num{0.00}\\
 & (\num{0.05}) & (\num{0.04})\\
\midrule
$HW_{i} \times Hispanic_{i} - WH_{i} \times Hispanic_{i}$ & -0.03 & -0.04\\
 & (0.07) & (0.07)\\
\midrule
Observations & \num{129359} & \num{129359}\\
\textit{Controlling for:} &  & \\
Hours Worked & X & X\\
Age & X & X\\
Year FE & X & X\\
Education &  & X\\
\bottomrule
\multicolumn{3}{l}{\rule{0pt}{1em}* p $<$ 0.1, ** p $<$ 0.05, *** p $<$ 0.01}\\
\end{tabular}
\begin{tablenotes}
\item[1] \footnotesize{This table includes the estimation results of equation (\ref{eq:iden}).}
\item[2] \footnotesize{The four groups stand for White Husband White Wife (WW), White Husband Hispanic Wife (WH), Hispanic Husband White (HW), and Hispanic Husband Hispanic Wife (HH). $Hispanic_{i}$ is a dummy variable equal to one if a person identifies as Hispanic and zero otherwise.}
\item[3] \footnotesize{The sample is restricted to men working full-time full-year and are waged and salaried workers.}
\item[4] \footnotesize{Column one has the regression results when controlling for hours worked, age, and years of fixed effects. Column two has the results after controlling for education.}
\end{tablenotes}
\end{threeparttable}}
\end{table}


% \newpage

% \begin{table}[H]
\centering\centering
\caption{Descriptive Statistics About Selection Into Occupation by Type \label{tab:occ-sum}}
\centering
\resizebox{\ifdim\width>\linewidth\linewidth\else\width\fi}{!}{
\begin{threeparttable}
\begin{tabular}[t]{>{}lcccc}
\toprule
\multicolumn{1}{c}{ } & \multicolumn{1}{c}{Hispanic Last Name} & \multicolumn{1}{c}{White Last Name} & \multicolumn{2}{c}{Difference-in-means: HW - WH} \\
\cmidrule(l{3pt}r{3pt}){2-2} \cmidrule(l{3pt}r{3pt}){3-3} \cmidrule(l{3pt}r{3pt}){4-5}
Occupation & Hispanic-White & White-Hispanic & Difference & p value\\
\midrule
\textbf{Management and Business} & 0.111 & 0.129 & -0.018 & 0.000\\
\textbf{STEM Occupations} & 0.030 & 0.040 & -0.009 & 0.000\\
\textbf{Healthcare Occupations} & 0.063 & 0.058 & 0.005 & 0.002\\
\textbf{Education and Social Services} & 0.062 & 0.064 & -0.002 & 0.237\\
\textbf{Arts, Media, and Entertainment} & 0.028 & 0.034 & -0.006 & 0.000\\
\addlinespace
\textbf{Service Occupations} & 0.107 & 0.099 & 0.008 & 0.000\\
\textbf{Manual and Industrial Labor} & 0.199 & 0.171 & 0.028 & 0.000\\
\bottomrule
\end{tabular}
\begin{tablenotes}
\item[1] Source: Current Population Survey (CPS) 1994-2019.
\end{tablenotes}
\end{threeparttable}}
\end{table}


\restoregeometry



%%%%%%%%%%%%%%%%%%%%%%%%%%%%%%%%%%%%
% PART V. Appendix
%%%%%%%%%%%%%%%%%%%%%%%%%%%%%%%%%%%%
\appendix
\begin{refsection}

% reset page numbers
\pagenumbering{arabic}
\setcounter{page}{1}
\end{refsection}

% %%%%%%%%%%%%%%%%%%%%%%%%%%%%%%%%%%%%
% % PART VI. Responses to Editor and Referees
% %%%%%%%%%%%%%%%%%%%%%%%%%%%%%%%%%%%%
\begin{refsection}
% Reset section numbering
\setcounter{section}{0}
\renewcommand{\thesection}{\Alph{section}}
\renewcommand{\thesubsection}{\Alph{section}.\arabic{subsection}}
\renewcommand{\thesubsubsection}{\Alph{section}.\arabic{subsection}.\arabic{subsubsection}}

        \section{Responses to Editors and Referee} \label{r&r:responses}
        \begin{spacing}{\rrxspc}
            I would like to thank the editor and the anonymous referees for their insightful comments, suggestions, and effort and time in reviewing this paper. I have addressed all the comments and suggestions in the revised manuscript. Below, I provide a summary of the changes made to the manuscript in response to the comments and suggestions.
        \end{spacing}

    \newpage
    
\section{Responses to Referee One}
I would like to thank referee one for the insightful comments and suggestions. Below is a detailed response to the comments and suggestions.
\begin{spacing}{\rrquote}
\begin{quotation}
\textbf{R1: } 1. My first concern relates to the threats to identification. It is feasible that children born to HW couples may differ systematically to children born to WH couples, particularly with respect to unobservable characteristics important for labor market outcomes. For example, HW pairs may exhibit differing parental characteristics related to parenting styles, parental preferences for education, gender norms and beliefs. The paper would be improved by including an expanded discussion around the threat to identification. In light of these issues, the paper would be improved by providing a more thorough discussion of the relative advantages of the current approach, compared to the conventional Oaxaca-Blinder decomposition.
\end{quotation}
\end{spacing}
\begin{spacing}{\rrxspc}
    Thank you for raising these concerns. I have significantly expanded the discussion of identification threats and the advantages of my approach compared to traditional methods. In Section \ref{sec:hw-wh-couples-data} (``From the Data: The Differences Between HW and WH Couples''), I now explicitly acknowledge that ``Unlike traditional decomposition methods, this comparison directly isolates the role of surname signaling, reducing the confounding effect of background disparities. By focusing specifically on children from interethnic marriages with similar parental characteristics but different surname ethnicities, I use a more targeted comparison group that better isolates the effect of perceived ethnicity from family background. Traditional decomposition approaches would struggle to separate discrimination effects from the broader socioeconomic gaps between Hispanic and White families overall. The smaller educational and income disparities between HW and WH families compared to HH and WW families, allowing for a clearer attribution of outcome differences to discrimination rather than unobserved heterogeneity. Nevertheless, I acknowledge that residual selection bias remains a concern, particularly regarding cultural and attitudinal factors that may systematically differ between HW and WH households despite similar educational and economic outcomes.''
\end{spacing}

\begin{spacing}{\rrxspc}
    However, I argue that this approach offers distinct advantages over traditional Oaxaca-Blinder-Kitagawa decomposition methods. First, by focusing specifically on children from interethnic marriages with similar parental characteristics but different surname ethnicities, I use a more targeted comparison group that better isolates the effect of perceived ethnicity from family background. Traditional decomposition approaches would struggle to separate discrimination effects from the broader socioeconomic gaps between Hispanic and White families overall. Second, the smaller educational and income disparities between HW and WH families compared to HH and WW families, allowing for a clearer attribution of outcome differences to discrimination rather than unobserved heterogeneity. 
\end{spacing}

\begin{spacing}{\rrxspc}
    The empirical evidence supports this approach: Table \ref{tab:synth} shows that interethnic couples are much more similar to each other than endogamous couples are to each other, which is exactly why comparing children of interethnic couples provides a better comparison than comparing children of endogamous marriages. Nevertheless, I acknowledge that residual selection bias remains a concern, particularly regarding cultural and attitudinal factors that may systematically differ between HW and WH households despite similar educational and economic outcomes. Moreover, it is likely that the parenting and cultural backgrounds of interethnic children are more similar to each other than those of children of endogamous marriages.
    \end{spacing}

\begin{spacing}{\rrxspc}
\begin{quotation}
    \textbf{R1: } 2. The analysis sample is restricted to individuals who self-identify as 'White'. It is likely that this self-identification of ethnicity is endogenous to labor market outcomes. Importantly, the unobservable determinants of racial identification are potentially correlated with labor market outcomes. The paper would be improved by including a more thorough discussion of this (sample) selection issue. Related to this is the depiction of parents born in the United States as 'White'. This seems particularly over-simplifying and ignores the long history of immigration. A more accurate description would be to refer to this group as US born.
    \end{quotation}
    \end{spacing}
    
\begin{spacing}{\rrxspc}
    Thank you for your comments and suggestions. I have addressed both concerns raised by the reviewer. First, regarding the endogeneity of self-identification, I now explicitly discuss this issue in the data section. I explain that ``The sample is restricted to Hispanic and non-Hispanic White individuals to avoid confounding racial factors.'' I argue that including those who do not identify as White could contaminate the estimate of bias against Hispanics due to racial signals. Moreover, as shown by \textcite{hadah2024effect}, there exists strong correlation between bias against Hispanics and self-reported Hispanic identity, and excluding Hispanics who experience ethnic attrition could lead to overestimated bias in the most discriminatory states. To address the comment on the long history of immigration in the US, I updated the manuscript to make clear that the interethnic children I am studying are US born. I also added the following: ``This study's focus on US-born children with foreign-born parents means the findings may not generalize to Hispanic children with US-born parents, who likely face different socioeconomic and cultural circumstances. The analysis does not account for heterogeneity in immigrant characteristics in different Spanish-speaking countries of origin, such as variations in educational attainment, socioeconomic status, and gender-specific migration patterns, which could influence both parental selection in migration and subsequent child outcomes.''
\end{spacing}    

\begin{spacing}{\rrxspc}
    Second, regarding the characterization of US-born parents, I have revised the language throughout the manuscript to make it clear that the analysis focuses on US-born individuals. I now explicitly address this concern: ``While the United States has a long history of immigration, the probability that a US-born parent in this sample is a second-generation or later immigrant from a Spanish-speaking country is very low, as shown in Table \ref{tab:mat3}. The majority of Hispanics in the US during this period were first- and second-generation immigrants. Only 3\% of native-born Americans identified as Hispanic during this period, making it statistically unlikely that an interethnic child with a native-born parent is also a second-generation Hispanic immigrant.'' I have also changed terminology throughout to use ``US-born'' rather than ``White'' when referring to native-born parents, and I clarify that this approach ``provides a cleaner identification of surname effects, which is the primary focus of this study.''
\end{spacing}

\begin{spacing}{\rrxspc}
    \begin{quotation}
        \textbf{R1: } 3. The discussion on the construction of the 'synthetic parents' is somewhat brief. The paper suggests that the potential parents are matched using the birth year of the child and the parent's place of birth to the children's information collected in the CPS sample. While not explicitly stated in the paper, I assume that this sample of synthetic parents is used to construct mean parental education and family income at the time of the birth of the child. The process for matching the potential parents to the children is somewhat aggregated and their is likely considerable heterogeneity in the mean educational attainment of the potential parents within the set of children with the same birth year with the same parent's country of birth. The paper could be improved by exploring the possibility of improving the quality of the match of potential parents by including further characteristics to match the potential parents to their children.
\end{quotation}
    \end{spacing}
        
\begin{spacing}{\rrxspc}
        Thank you for raising this point. I have significantly expanded the discussion of synthetic parents construction and provided a detailed concrete example. In the data section, I now include: ``To illustrate the construction of synthetic parents more concretely, consider someone who was 35 years old in 1999, meaning they were born in 1964. If this person's mother was born in Mexico and their father was born in the United States, their 'synthetic parents' would be identified using the 1970 Census data, when the person was 6 years old. The 'synthetic mother' would have the average characteristics (education, income, etc.) of Mexican-born women who were married to US-born men and had children around 1964, when they were between 20 and 35 years old (meaning they were born between 1929 and 1944). Similarly, the 'synthetic father' would have the average characteristics of US-born men who were married to Mexican-born women and had children in that same year.''

        I acknowledge the reviewer's suggestion about including additional characteristics to improve the matching quality. While incorporating more matching variables (such as parent's age at birth) could potentially reduce heterogeneity within the synthetic parent groups, the current approach balances precision with sample size considerations. Adding more matching criteria would create increasingly small cells in the Census data, potentially leading to unstable estimates or empty cells, particularly for less common parent birthplace combinations. Moreover, since the synthetic parent characteristics serve as controls rather than the primary variables of interest, the current level of aggregation is sufficient for the paper's main empirical strategy. 
\end{spacing}

\begin{spacing}{\rrxspc}
    I also acknowledge the limitations: ``This aggregation process introduces some heterogeneity, as individual parents vary considerably from these group-level measures. However, this approach provides valuable information about typical family background characteristics that would otherwise be completely unobserved. The synthetic parents' characteristics serve as useful proxies for the socioeconomic environment in which these children were raised, allowing me to control for important background factors that influence educational and labor market outcomes. While more refined matching would be ideal, the publicly available data limit the potential matching dimensions. Nevertheless, the current approach provides meaningful estimates while acknowledging these data constraints.''
\end{spacing}

\begin{spacing}{\rrxspc}
    \begin{quotation}
        \textbf{R1: } 4. The description of the estimated model is somewhat brief. However, there are two main issues associated with statistical inference on the estimated parameters in model (1). First, there is a 'generated regressor' issue associated with using estimated group-level parental education. A failure to account for this sampling variation will lead to misleadingly small standard errors. Second, given this group structure for parental education, it seems reasonable to assume that the model errors are uncorrelated across clusters but correlated within (potential parents) clusters. It is well understood that failing to account for this 'clustering' problem can lead to misleadingly small standard errors, narrow confidence intervals, and low p-values.
    \end{quotation}
\end{spacing}

\begin{spacing}{\rrxspc}
    I am grateful to the referee for raising this important concern. I have expanded the discussion of these methodological issues in the empirical approach section. Regarding the generated regressor issue: ``The parental characteristics are included solely as controls and not for inference purposes. I do not interpret these coefficients, as my primary interest lies in $\beta_1$, which captures the gap between Hispanic and White last names. Since I am only using these synthetic characteristics as controls and not making inference about their coefficients, this should not affect the standard errors or p-values of my coefficient of interest $\beta_1$.''

    Regarding the clustering concern: To address this issue, I re-estimated the log annual earnings model using standard errors clustered at the parental birthplace level (mother's and father's birthplace), which corresponds to the level at which `synthetic' parental characteristics were estimated---see table below. The clustered standard errors for $\beta_1$ became slightly smaller (ranging from 0.00 to 0.01 across specifications) compared to the original specification. The fact that standard errors decreased rather than increased after clustering indicates that the original standard errors were, if anything, conservative for my main coefficient of interest. Moreover, since immigrants from Spanish-speaking countries have strong heterogeneous geographical preferences on where to live, clustering at the state level would further address the concern raised by the referee. For example, Cuban immigrants strongly select Florida when they move to the US, while Mexican immigrants self-select to California and Texas. This geographical clustering pattern actually strengthens the case for state-level clustering, as it directly addresses the referee's concern about spatial correlation in the error term. By clustering standard errors at the state level, I account for this within-state correlation that arises from the non-random geographical distribution of immigrant groups.
    
\end{spacing}

\newgeometry{bottom=1.0in}
\begin{table}[H]
\centering\centering
\caption*{Effect of Having Hispanic Last Name (Log Annual Earnings)}
\centering
\begin{threeparttable}
\begin{tabular}[t]{lccccc}
\toprule
  & \specialcell{(1) \\ Log annual \\ earnings} & \specialcell{(2) \\ Log annual \\ earnings} & \specialcell{(3) \\ Log annual \\ earnings} & \specialcell{(4) \\  Log annual \\ earnings} & \specialcell{(5) \\  Log annual \\ earnings}\\
\midrule
$HW_{ist}$ & \num{-0.05}*** & \num{-0.05}*** & \num{-0.05}*** & \num{-0.02}*** & \num{-0.01}***\\
 & (\num{0.00}) & (\num{0.00}) & (\num{0.01}) & (\num{0.00}) & (\num{0.00})\\
Constant & \num{10.42}*** & \num{9.46}*** &  &  & \\
 & (\num{0.00}) & (\num{0.02}) &  &  & \\
\midrule
\textit{Controlling for:} &  &  &  &  & \\
Hours Worked &  & X & X & X & X\\
Age &  &  & X & X & X\\
State FE &  &  & X & X & X\\
Year FE &  &  & X & X & X\\
State-Year FE &  &  & X & X & X\\
Education &  &  &  & X & X\\
Parental Background &  &  &  &  & X\\
Observations & \num{3621} & \num{3621} & \num{3621} & \num{3621} & \num{3621}\\
\bottomrule
\multicolumn{6}{l}{\rule{0pt}{1em}* p $<$ 0.1, ** p $<$ 0.05, *** p $<$ 0.01}\\
\end{tabular}
\begin{tablenotes}
\item[1] {\setstretch{1.0}\footnotesize{This table includes the estimation results of equation (\ref{eq:1a}).}}
\item[2] {\setstretch{1.0}\footnotesize{HW is an indicator variable that is equal to 1 if a person is the child of a Hispanic-father and White-mother.}}
\item[3] {\setstretch{1.0}\footnotesize{The sample is restricted to men working full-time full-year and are wage and salary workers.}}
\item[4] {\setstretch{1.0}\footnotesize{Column one has the regression results when controlling for hours worked, age, education, year and state fixed effects. Column two has the results after controlling for education.}}
\item[5] {\setstretch{1.0}\footnotesize{Standard errors are clustered at the mother's and father's birthplace level.}}
\end{tablenotes}
\end{threeparttable}
\end{table}

\restoregeometry

\begin{spacing}{\rrxspc}
\begin{quotation}
    \textbf{R1: } 5. The paper would be improved through an expanded discussion of the impact of measurement error on the reported estimates. While there is a well understood result in the measurement error literature that measures of group mean parental education will provide estimates that are more robust to the presence of measurement error in individual level measures of parental education, there is still an issue with the non-random attrition of potential parents. Specifically, not all potential parents have children in the CPS sample, nor do all children in the CPS sample have parents in the sample of potential parents. Moreover, the non-random attrition of potential parents implies that mean parental education may be systematically higher or lower than the actual parental education of the children in the CPS sample. The paper would be improved by providing an expanded discussion of the likely impacts of measurement error on the reported estimates.
    \end{quotation}
\end{spacing}
\begin{spacing}{\rrxspc}
    Thank you for your comments. I have clarified this methodological point and expanded the discussion of measurement error concerns. First, I want to clarify that the concern about attrition does not apply to my analysis: The synthetic parents are constructed using the birth year of the child and are not based on actual parents in the CPS sample at the time of the survey. The place of birth of parents---mothers and fathers---are questions asked of all participants in the Current Population Survey (CPS) starting 1994. Therefore, I do not use the information of parents who are in the CPS sample at the time of the survey, and I construct synthetic parents from Census data. Because this approach uses Census data rather than contemporaneous parent-child pairs, concerns about non-random attrition of parents from the CPS sample do not apply to my analysis. 
    \end{spacing} 

\begin{spacing}{\rrxspc}     
    Regarding measurement error more broadly, I have expanded the discussion in the empirical approach section: ``A potential concern is measurement error in using CPS data to infer parental ethnicity, since it relies on place of birth rather than self-identified Hispanic origin. However, during the period studied, second-generation Hispanics were rare: only about 3\% of native-born Americans identified as Hispanic (see Table \ref{tab:mat3}, which presents Hispanic identification rates by nativity). This makes it unlikely that US-born parents in interethnic unions are second-generation+ Hispanic immigrants.''
    \end{spacing}
\begin{spacing}{\rrxspc}
    Additionally, while using census averages for synthetic parent characteristics does introduce some imprecision compared to actual parent data, this approach avoids the selection bias that would arise from using only parent-child pairs who remain in the CPS sample. The synthetic parents methodology, used by \textcite{rubinstein2014pride}, provides population-level averages that are more representative than the selected sample of parents who happen to be surveyed alongside their children. Moreover, since these synthetic parent controls are not interpreted causally, any measurement error from using group means rather than individual-level data is unlikely to bias the estimated surname effect, which is the key parameter of interest.
    \end{spacing}

\clearpage
\pagebreak

    \section{Responses to Referee Two}
    I would like to thank referee two for the insightful and constructive comments and suggestions. Below is a detailed response to the comments and suggestions.

    \begin{spacing}{\rrquote}
        \begin{quotation}
        \textbf{R2: } 1. The paper needs to be more clearly motivated in the introduction. There are a lot of reasons why studying discrimination by race and ethnicity is important and the paper would benefit from clearly articulating this, including discussing the implications of this discrimination. You mention economic mobility and I think there is more you can say about this. I also think clearly stating the contributions of the paper earlier on would be important.
        \end{quotation}
        \end{spacing}
        
        \begin{spacing}{\rrxspc}
            Thank you for the important suggestion. I have expanded the introduction to better motivate the study. I have included a discussion of the implications of discrimination on economic mobility and the contributions of the paper. I have also included a discussion of the importance of studying discrimination    
    \end{spacing}

    \begin{spacing}{\rrquote}
        \begin{quotation}
        \textbf{R2: } 2. I would also like to see a clearer theory and review of the literature on the topic. For example, why should we expect that discrimination will affect educational attainment? Hispanics are one of the fastest growing groups entering college, but the extent to which they complete an associate or BA degree may reflect discrimination that happens when students are in school that ultimately translates into differences in attainment. Why might we expect differences in employment and years of education?
        Why might we expect differences by gender?
        \end{quotation}
        \end{spacing}
        
        \begin{spacing}{\rrxspc}
            I appreciate this suggestion. Here are my replies to the different comments.
            
            First, regarding including more on why discrimination could affect educational attainment. I have expanded the part of the introduction that discusses differences in education and the channels in which discrimination can affect educational attainment. The literature on discrimination and education shows that discrimination can affect educational attainment of minorities through various channels, including differences in school quality and bias from teachers, administrators, counselors, etc. These biases could lead to differences in educational attainment by preventing students access to some schools, recommendation letters, or counselors restricting access to more advanced courses, etc. Consequently, these biases could lead to differences in educational attainment. For example, having access to advanced courses in high school or better recommendation letters could affect the likelihood of attending college.

            Second, if minorities face discrimination in access to education and the labor market, then we would expect differences in employment and years of education. The literature shows that discrimination can affect labor market outcomes through various channels, including differences in access to jobs and wage differences.
            
            Finally, to address the comment here, and in other places, on why we might expect differences between genders. I added a discussion of why we might expect differences between men and women in the results section. I believe that showing that there might be heterogeneity in gaps between men and women in educational outcomes to be an important contribution of the paper since the literature studying gaps in earnings mainly focuses on the average gap between men. Moreover, showing that couples with a Hispanic husband and a White wife do not invest differently in their children than couples with a White husband and a Hispanic wife could be a way to test for cultural differences between the two groups. 
        \end{spacing}

    \begin{spacing}{\rrquote}
        \begin{quotation}
        \textbf{R2: } 3. Occupational segregation can also reflect discrimination in the labor market with important implications for economic mobility and other outcomes, and it would be important to mention this in the literature review. In general, the review of the literature needs to provide more details about the different studies and how your paper contributes to that literature above and beyond using a cleaner comparison.
        \end{quotation}
        \end{spacing}
        
        \begin{spacing}{\rrxspc}
            Thank you for this suggestion. I have included a discussion of occupational segregation in the literature review and added how my paper contributes to the different strands of the literature.
    \end{spacing}

    \begin{spacing}{\rrquote}
        \begin{quotation}
        \textbf{R2: } 4. There are places throughout the paper where additional references are needed. For example, when you state in the introduction on page 3 that discrimination can lead to lower wages, reduced opportunities, and hinder assimilation this statement needs references.
        \end{quotation}
        \end{spacing}
        
        \begin{spacing}{\rrxspc}
            Thank you for pointing this out. I have added references to such statements. 
    \end{spacing}

    \begin{spacing}{\rrquote}
        \begin{quotation}
        \textbf{R2: } 5. Can you explain why you discuss assimilation and in what ways this is connected to your theory given that you are focusing on U.S. born children? If this matters because you are focusing on U.S. born Hispanic children with one foreign-born parent, then you need to clearly state this.
        \end{quotation}
        \end{spacing}
        
        \begin{spacing}{\rrxspc}
            Thank you for the comment. I added a footnote to the introduction to address it. 
    \end{spacing}

    \begin{spacing}{\rrquote}
        \begin{quotation}
        \textbf{R2: } 6. You also need to state clearly and early in the paper that your study focuses on children with a foreign-born parent. Children with U.S. born Hispanic parents may be different than those with foreign-born parents and you could more explicitly discuss this in the paper.
        \end{quotation}
        \end{spacing}
        
        \begin{spacing}{\rrxspc}
            I appreciate this comment. I made it more clear throughout the paper that I focus on children with a foreign-born parent. 
    \end{spacing}

    \begin{spacing}{\rrquote}
        \begin{quotation}
        \textbf{R2: } 7. Also given the differences in the characteristics of immigrants from different Spanish-speaking countries living in the U.S. (differences in socioeconomic status, education, etc.), you probably want to mention that your study is not capturing this. Further, to the extent that men and women migrate from different countries and have different pre- and post-migration characteristics then this might affect your results.

        One possible way to address this is to conduct sensitivity analyses limited to Children whose parents were likely born in Mexico or who respond themselves that they are Mexican. Given that immigrants from Mexico are the largest immigrant group from a Spanish speaking country in the U.S., this may offer an even cleaner comparison. There could still be differences in the characteristics of Mexican mothers and fathers, but you could potentially check this in the data.        
        \end{quotation}
        \end{spacing}
        
        \begin{spacing}{\rrxspc}
            Thank you for this suggestion. I made it clear in the paper that I do not capture differences in the characteristics of immigrants from different Spanish-speaking countries. I also added some results and discussion as a sensitivity analysis breaking down the results for Hispanic children with Mexican parents versus Hispanic children with non-Mexican parents. 
    \end{spacing}

    \begin{spacing}{\rrquote}
        \begin{quotation}
        \textbf{R2: } 8. There are several sections that are repeated in the paper and I would suggest you streamline the text. For example, you review results twice, but this isn’t necessary.

        Similarly, I think you can more systematically organize the section describing your empirical strategy and clearly explaining your identification strategy, the concerns that it helps you overcome, your assumptions, and how you are testing whether these assumptions likely hold. Currently this is explained in multiple sections throughout the paper.        
        \end{quotation}
        \end{spacing}
        
        \begin{spacing}{\rrxspc}
            I appreciate this comment. I have streamlined the text and reorganized the paper to address these concerns. 
    \end{spacing}

    \begin{spacing}{\rrquote}
        \begin{quotation}
        \textbf{R2: } 9. Can you provide an example that illustrates how you link individuals in the CPS to the synthetic parents?
              
        \end{quotation}
        \end{spacing}
            I added an example to the data section that illustrates how I link individuals in the CPS to the synthetic parents.
        \begin{spacing}{\rrxspc}
            
    \end{spacing}

    \begin{spacing}{\rrquote}
        \begin{quotation}
        \textbf{R2: } 10. I would like more details about the sample and the decisions you make. For example, you identify your sample as U.S. born children who identify as White in the CPS but some of these may also identify as Hispanic no? Do you restrict your sample to respondents who identify as White Hispanic or not? How many individuals in your sample identify as White non-Hispanic even if they have a parent born in a Spanish speaking country and how many do not? If someone has a Hispanic parent but do not identify as Hispanic those may be different than people with a Hispanic parent who identify as Hispanic.
        
        If there are people who identify as White non Hispanic in your sample even though they have a parent born in a Spanish speaking country, can you do a sensitivity analysis removing them?

        For example, on page 18 you say “I also find a significant earnings gap between those that identify as Hispanic.” Can you explain this statement? Is this related to the point I made above?

        \end{quotation}
        \end{spacing}
        
        \begin{spacing}{\rrxspc}
            Thank you for the comment. I added more description of the sample to the data section and the reasoning of why I chose those who self-identify as White. 
    \end{spacing}

    \begin{spacing}{\rrquote}
        \begin{quotation}
        \textbf{R2: } 11. If there are people who identify as White non Hispanic in your sample even though they have a parent born in a Spanish speaking country, can you do a sensitivity analysis removing them?

        \end{quotation}
        \end{spacing}
        
        \begin{spacing}{\rrxspc}
            Thank you for the comment. I added more description of the sample to the data section and the reasoning of why I chose those who self-identify as White. 
    \end{spacing}

    \begin{spacing}{\rrquote}
        \begin{quotation}
        \textbf{R2: } 12. For example, on page 18 you say “I also find a significant earnings gap between those that identify as Hispanic.” Can you explain this statement? Is this related to the point I made above?

        \end{quotation}
        \end{spacing}
        
        \begin{spacing}{\rrxspc}
            Thank you for the comment. I added more description of the sample to the data section and the reasoning of why I chose those who self-identify as White. 
    \end{spacing}

    \begin{spacing}{\rrquote}
        \begin{quotation}
        \textbf{R2: } 13. You report results for men and women and, I am sorry if I missed it, but I would like to see this motivated in the paper as there are many reasons for doing the analyses separately. See my earlier point on the lit review/theory.
              
        \end{quotation}
        \end{spacing}
            Please see my response to the earlier comment on why we might expect differences between men and women.
        \begin{spacing}{\rrxspc}
            
    \end{spacing}

    \begin{spacing}{\rrquote}
        \begin{quotation}
        \textbf{R2: } 14. It would also be important to discuss effect sizes. Some of the findings seem to be small and you should discuss whether they are economically meaningful. Further, there should be more discussion about what your findings mean to understand discrimination for Hispanics and how it plays out, including by gender and in what ways they fall short in answering this question.
              
        \end{quotation}
        \end{spacing}
        
        \begin{spacing}{\rrxspc}
            Thank you for the suggestion. I have included a discussion of the effect sizes and the economic meaning of the results. I have also included a discussion of what the findings mean for understanding discrimination against Hispanics and how it plays out.
    \end{spacing}

    \begin{spacing}{\rrquote}
        \begin{quotation}
        \textbf{R2: } 15. How are you measuring the different outcomes? I don’t believe you discuss in the paper. This may be obvious, but I think it is still important to mention for clarity.
              
        \end{quotation}
        \end{spacing}
    
        \begin{spacing}{\rrxspc}
            Thank you for the comment. I have included a discussion of how I measure the different outcomes in the data section.        
    \end{spacing}

    \begin{spacing}{\rrquote}
        \begin{quotation}
        \textbf{R2: } 16. When you control for education in models that examine earnings, the effect becomes statistically insignificant, what happens if you include industry or occupation fixed effects? Can you do that? Then you'd be comparing people within the same industry or occupation.
              
        \end{quotation}
        \end{spacing}
            
        \begin{spacing}{\rrxspc}
            I appreciate this comment. I have added a discussion of the results when I include occupation fixed effects to the sensitivity analysis section.
    \end{spacing}

    \begin{spacing}{\rrquote}
        \begin{quotation}
        \textbf{R2: } 17. The conclusion repeats much of what was said in the body of the paper, and I would like to see more discussion about the implications of your findings to understand disparities in education and labor outcomes between people with different ethnic background, and to what extent you can conclude these differences can be attributed to discrimination. It would also be important to compare your results with other literature on this topic.
              
        \end{quotation}
        \end{spacing}
        
        \begin{spacing}{\rrxspc}
            Thank you so much for this comment. I have expanded the conclusion to include more discussion about the implications of the findings and how they can be attributed to discrimination. I have also included a discussion of how my results compare to the literature on this topic.
    \end{spacing}

    \begin{spacing}{\rrquote}
        \begin{quotation}
        \textbf{R2: } 18. On page 14, this sentence seems to be incomplete: “Consequently, comparing WH and HW children to each other to analyze discrimination against Hispanics in the labor market.”

              
        \end{quotation}
        \end{spacing}
        
        \begin{spacing}{\rrxspc}
            Thank you for pointing this out. I have corrected this sentence.
    \end{spacing}

    \begin{spacing}{\rrquote}
        \begin{quotation}
        \textbf{R2: } 19. Throughout the paper you refer to people with a parent born in a Spanish speaking country as “children who have a Spanish-sounding last name” I would encourage you to soften this language and say “who likely have” a Spanish sounding last name. First, you do not know if a person actually does have the last name of their father. Second, while it is true that Spanish sounding last names are very common in Spanish-speaking Latin American countries and among Latinos in the U.S., this may not be true across the board and many people born in Spanish speaking countries, who are Spanish speakers and Hispanic, may not have a traditionally sounding Spanish last name.
    
        \end{quotation}
        \end{spacing}
        
        \begin{spacing}{\rrxspc}
            Thank you for your comment. I changed the mention of "children who have a Spanish-sounding last name" to "children who likely have a Spanish-sounding last name" throughout the paper.
    \end{spacing}
    \end{refsection}

    \clearpage
    \pagebreak

    \section{Responses to Referee three}
    I would like to thank referee three the comments. Below is a detailed response to the comments and suggestions.

    \begin{spacing}{\rrquote}
        \begin{quotation}
        \textbf{R3: } 1. As the author states on page 4, the key identifying assumption is that people born to HW parents are similar to their WH peers. Only when this assumption is fulfilled can the author attribute the estimated coefficients to evidence of discrimination. However, Table 4 clearly shows that the differences between HW and WH synthetic parents’ characteristics, such as the father’s/mother’s education and total family income, are all significantly different. More specifically, HW families exhibit lower levels of education and income compared to their WH counterparts. Based on this, it is not clear whether the estimated coefficients are indeed evidence of discrimination or they merely reflect the fact that HW children grew up in an environment with less resources/educated parents.

        The manuscript heavily adopts the methodology of Rubinstein and Brenner (2014), which examines the impacts of having a Sephardic-sounding surname on wages by comparing the Israeli-Jewish men born to Sephardic fathers and Ashkenazi mothers (SA) with those born to Ashkenazi fathers and Sephardic mothers (AS). They too find that the AS and SA families have statistically different educational and labor market outcomes. However, in their case, despite the fact that Sephardic Israelis face tougher labor market conditions, SA parents exhibit better education and labor market outcomes than AS parents. Therefore, any evidence indicating that SA offspring have worse labor market outcomes provides convincing (and potentially a lower bound) evidence of discrimination.

        \end{quotation}
        \end{spacing}
        
        \begin{spacing}{\rrxspc}
            Thank you for the comment. Even though the synthetic parents have different characteristics, the estimation strategy rests on the fact that selection in the marriage market decreases the differences between HW and WH couples. This is the reason why comparing children of intermarried couples provides a better comparison than comparing children of endogamous marriages, i.e. marriages where both parents are either non-Hispanic White or Hispanic. I also included a discussion of the relative advantages of the current approach compared to the conventional Oaxaca-Blinder-Kitagawa decomposition. I argue that the current approach provides a more accurate estimate of gaps that are due to discrimination. In fact, the concern the reviewer raises is one of the reasons why my approach is preferred over the Oaxaca-Blinder-Kitagawa since children of endogamous marriages are more likely to have different characteristics than children of intermarried couples, including those that are unobservable. I argue, and show from the data, that children of intermarried couples are more likely to have similar characteristics than children of endogamous couples.
    \end{spacing}

\clearpage
\pagebreak
%%%%%%%%%%%%%%%%%%%%%%%%%%%%%%%%%%%%
% PART VII. Online Appendix
%%%%%%%%%%%%%%%%%%%%%%%%%%%%%%%%%%%%
\begin{refsection}
% Reset section numbering
\setcounter{section}{0}
\renewcommand{\thesection}{\Alph{section}}
\renewcommand{\thesubsection}{\Alph{section}.\arabic{subsection}}
\renewcommand{\thesubsubsection}{\Alph{section}.\arabic{subsection}.\arabic{subsubsection}}
\setcounter{page}{1}

% Title Page for Appendix
% special footnote symbols
\renewcommand*{\thefootnote}{\fnsymbol{footnote}}
\begingroup
\doublespacing
\centering
\Large ONLINE APPENDIX \\[1.5em]
\LARGE \PAPERTITLE \\[0.75em]
\large
\AUTHORHADAH\footnote[1]{\AUTHORHADAHINFO}\\[1.0em]
\endgroup
\clearpage

% Online appendix
%%%%%%%%%%%%%%%%%%%%%%%%%%%%%%%%%%%%
% Appendix A, Solution and Estimation Details
%%%%%%%%%%%%%%%%%%%%%%%%%%%%%%%%%%%%

% Set equation, figure, table indexing
\renewcommand{\thefigure}{A.\arabic{figure}}
\setcounter{figure}{0}
\renewcommand{\thetable}{A.\arabic{table}}
\setcounter{table}{0}
\renewcommand{\theequation}{A.\arabic{equation}}
\setcounter{equation}{0}
\renewcommand{\thefootnote}{A.\arabic{footnote}}
\setcounter{footnote}{0}

\section{Tables}\label{appendix:figs}


\newpage
\newgeometry{bottom=1.0in}

\begin{table}[H]
\tablefont
\caption{Summary statistics of synthetic parents by couple type (Mexican Hispanics) \label{tab:synthmex}}
\centering
\resizebox{\linewidth}{!}{
\begin{threeparttable}
\begin{tabular}[t]{>{\raggedright\arraybackslash}p{5cm}cccccc}
\toprule
\multicolumn{1}{c}{ } & \multicolumn{4}{c}{Father's and Mother's Ethnicities} & \multicolumn{2}{c}{Differences} \\
\cmidrule(l{3pt}r{3pt}){2-5} \cmidrule(l{3pt}r{3pt}){6-7}
Variables & \specialcell{White \\ White \\ (WW) \\ (1)} & \specialcell{White \\ Hispanic \\ (WH) \\ (2)} & \specialcell{Hispanic \\ White \\ (HW) \\ (3)} & \specialcell{Hispanic \\ Hispanic \\ (HH) \\ (4)} & \specialcell{HH - WW \\ (5)} & \specialcell{HW - WH \\ (6)}\\
\midrule
Husband's education (Total Years) & \specialcell{12.75\\(0.61)} & \specialcell{10.80\\(1.41)} & \specialcell{9.14\\(1.52)} & \specialcell{7.85\\(1.24)} & \specialcell{-4.90***\\(0.00)} & \specialcell{-1.65***\\(0.01)}\\
Wife's education (Total Years) & \specialcell{12.47\\(0.56)} & \specialcell{9.27\\(1.71)} & \specialcell{10.46\\(1.32)} & \specialcell{7.80\\(1.36)} & \specialcell{-4.67***\\(0.00)} & \specialcell{1.19***\\(0.00)}\\
Total Household education (Total Years) & \specialcell{25.22\\(1.17)} & \specialcell{20.06\\(3.09)} & \specialcell{19.60\\(2.80)} & \specialcell{15.65\\(2.60)} & \specialcell{-9.57***\\(0.00)} & \specialcell{-0.47**\\(0.01)}\\
Log Total Family Income & \specialcell{10.72\\(0.09)} & \specialcell{10.42\\(0.14)} & \specialcell{10.33\\(0.12)} & \specialcell{10.20\\(0.07)} & \specialcell{-0.52***\\(0.00)} & \specialcell{-0.09***\\(0.00)}\\
Fertility & \specialcell{3.77\\(0.40)} & \specialcell{4.26\\(0.64)} & \specialcell{4.33\\(0.64)} & \specialcell{4.46\\(0.49)} & \specialcell{0.69***\\(0.00)} & \specialcell{0.07***\\(0.00)}\\
\bottomrule
\end{tabular}
\begin{tablenotes}
\item[1] Source: The 1960-2000 Census for synthetic parents, and 1994-2019 Current Population Surveys (CPS) for children's outcomes
\item[2] The data is restricted to native-born United States citizens who are also White, between the ages of 25 and 40, and have kids. I identify the ethnicity of a person's parents through the parent's place of birth. A parent is Hispanic if they were born in a Mexico. A parent is White if they were born in the United States.
\end{tablenotes}
\end{threeparttable}}
\end{table}


\newpage

\begin{table}[H]

\caption{Summary Statistics of Synthetic Parents by Couple Type (Non-Mexican Hispanics) \label{tab:syntnonmex}}
\centering
\resizebox{\linewidth}{!}{
\begin{threeparttable}
\begin{tabular}[t]{>{\raggedright\arraybackslash}p{5cm}cccccc}
\toprule
\multicolumn{1}{c}{ } & \multicolumn{4}{c}{Father's and Mother's Ethnicities} & \multicolumn{2}{c}{Differences} \\
\cmidrule(l{3pt}r{3pt}){2-5} \cmidrule(l{3pt}r{3pt}){6-7}
Variables & \specialcell{White \\ White \\ (WW) \\ (1)} & \specialcell{White \\ Hispanic \\ (WH) \\ (2)} & \specialcell{Hispanic \\ White \\ (HW) \\ (3)} & \specialcell{Hispanic \\ Hispanic \\ (HH) \\ (4)} & \specialcell{HH - WW \\ (5)} & \specialcell{HW - WH \\ (6)}\\
\midrule
Husband's education (Total Years) & \specialcell{12.58\\(2.88)} & \specialcell{13.40\\(2.88)} & \specialcell{12.68\\(3.38)} & \specialcell{10.47\\(3.95)} & \specialcell{-2.11**\\(0.01)} & \specialcell{-0.72**\\(0.03)}\\
Wife's education (Total Years) & \specialcell{12.36\\(2.40)} & \specialcell{12.70\\(2.81)} & \specialcell{12.57\\(2.76)} & \specialcell{10.20\\(3.67)} & \specialcell{-2.16**\\(0.01)} & \specialcell{-0.14**\\(0.03)}\\
Total Household education (Total Years) & \specialcell{24.95\\(4.77)} & \specialcell{26.12\\(4.99)} & \specialcell{25.28\\(5.51)} & \specialcell{20.75\\(6.84)} & \specialcell{-4.20**\\(0.02)} & \specialcell{-0.84*\\(0.05)}\\
Log Total Family Income & \specialcell{10.75\\(0.57)} & \specialcell{10.84\\(0.60)} & \specialcell{10.82\\(0.62)} & \specialcell{10.51\\(0.66)} & \specialcell{-0.24***\\(0.00)} & \specialcell{-0.02***\\(0.01)}\\
Husband's Log Hourly Earnings & \specialcell{1.74\\(0.83)} & \specialcell{2.00\\(0.82)} & \specialcell{1.96\\(0.87)} & \specialcell{1.48\\(0.87)} & \specialcell{-0.27***\\(0.00)} & \specialcell{-0.04**\\(0.01)}\\
\addlinespace
Wife's Log Hourly Earnings & \specialcell{1.60\\(0.93)} & \specialcell{1.93\\(0.82)} & \specialcell{1.90\\(0.89)} & \specialcell{1.55\\(0.84)} & \specialcell{-0.05***\\(0.01)} & \specialcell{-0.03**\\(0.02)}\\
Fertility & \specialcell{3.84\\(1.44)} & \specialcell{3.52\\(1.28)} & \specialcell{3.66\\(1.42)} & \specialcell{3.95\\(1.62)} & \specialcell{0.10***\\(0.01)} & \specialcell{0.14**\\(0.02)}\\
\bottomrule
\end{tabular}
\begin{tablenotes}
\item[1] Source: The 1960-2000 Census for synthetic parents, and 1994-2019 Current Population Surveys (CPS) for children's outcomes
\item[2] The data is restricted to native-born United States citizens who are also White, between the ages of 25 and 40, and have kids. I identify the ethnicity of a person's parents through the parent's place of birth. A parent is Hispanic if they were born in a Spanish-speaking country other than Mexico. A parent is White if they were born in the United States.
\end{tablenotes}
\end{threeparttable}}
\end{table}


\newpage

\begin{table}[H]
\centering\centering
\caption{Effect of Having Hispanic Last Name (Log Weekly Earnings) \label{tab:lastnamereg-weekearm}}
\centering
\begin{threeparttable}
\begin{tabular}[t]{lccc}
\toprule
  & \specialcell{(1) \\ Log weekly \\ earnings} & \specialcell{(2) \\ Log weekly \\ earnings} & \specialcell{(3) \\  Log weekly \\ earnings}\\
\midrule
$HW_{ist}$ & \num{-0.04}*** & \num{-0.06}*** & \num{-0.03}**\\
 & (\num{0.01}) & (\num{0.01}) & (\num{0.01})\\
Constant & \num{6.65}*** &  & \\
 & (\num{0.03}) &  & \\
\midrule
\textit{Controlling for:} &  &  & \\
State FE &  & X & X\\
Year FE &  & X & X\\
State-Year FE &  & X & X\\
Age &  & X & X\\
Education &  &  & X\\
Observations & \num{7444} & \num{7444} & \num{7444}\\
\bottomrule
\multicolumn{4}{l}{\rule{0pt}{1em}* p $<$ 0.1, ** p $<$ 0.05, *** p $<$ 0.01}\\
\end{tabular}
\begin{tablenotes}
\item[1] {\setstretch{1.0}\footnotesize{This table includes the estimation results of equation (\ref{eq:1a}) where the dependent variable is log weekly earnings.}}
\item[2] {\setstretch{1.0}\footnotesize{HW is an indicator variable that is equal to 1 if a person is the child of a Hispanic-father and White-mother.}}
\item[3] {\setstretch{1.0}\footnotesize{The sample is restricted to men working full-time and are wage and salary workers.}}
\item[4] {\setstretch{1.0}\footnotesize{Column one has the regression results when controlling for hours worked, age, education, year and state fixed effects. Column two has the results after controlling for education.}}
\item[5] {\setstretch{1.0}\footnotesize{Standard errors are clustered on the state level.}}
\end{tablenotes}
\end{threeparttable}
\end{table}

\newpage

\begin{table}[H]
\centering\centering
\caption{Effect of Having Hispanic Last Name on Educational Outcomes: Hispanics with Mexican Ancestry \label{tab:lastname-ed-reg-mex}}
\centering
\begin{threeparttable}
\begin{tabular}[t]{lcccc}
\toprule
  & \specialcell{(1) \\ Years of \\ Education} & \specialcell{(2) \\ High School \\ Dropout} & \specialcell{(3) \\ Associate Degree} & \specialcell{(4) \\ Bachelor Degree}\\
\midrule
\addlinespace[0.5em]
\multicolumn{5}{l}{\textit{Panel A: Full Sample}}\\
\midrule \hspace{1em}$HW_{ist}$ & -0.31*** & 0.00 & -0.03*** & -0.04***\\
\hspace{1em} & (0.04) & (0.01) & (0.01) & (0.01)\\
\hspace{1em}Observations & 74926 & 76499 & 61273 & 76499\\
\addlinespace[0.5em]
\multicolumn{5}{l}{\textit{Panel B: Women}}\\
\midrule \hspace{1em}$HW_{ist}$ & -0.35*** & 0.01 & -0.03*** & -0.05***\\
\hspace{1em} & (0.05) & (0.01) & (0.01) & \vphantom{1} (0.01)\\
\hspace{1em}Observations & 39867 & 40608 & 31761 & 40608\\
\addlinespace[0.5em]
\multicolumn{5}{l}{\textit{Panel C: Men}}\\
\midrule \hspace{1em}$HW_{ist}$ & -0.28*** & -0.01 & -0.02*** & -0.04***\\
\hspace{1em} & (0.05) & (0.01) & (0.01) & (0.01)\\
\hspace{1em}Observations & 35059 & 35891 & 29512 & 35891\\
Full Sample's Mean & 13.15 & 0.46 & 0.14 & 0.2\\
Women's Mean & 13.26 & 0.46 & 0.15 & 0.22\\
Men's Mean & 13.03 & 0.45 & 0.13 & 0.18\\
\bottomrule
\multicolumn{5}{l}{\rule{0pt}{1em}* p $<$ 0.1, ** p $<$ 0.05, *** p $<$ 0.01}\\
\end{tabular}
\begin{tablenotes}
\item[1] {\setstretch{1.0}\footnotesize{This table includes the estimation results of equation (\ref{eq:1a}). All regressions include state-year fixed effects.}}
\item[2] {\setstretch{1.0}\footnotesize{HW is an indicator variable that is equal to 1 if a person is the child of a Hispanic-father and White-mother.}}
\item[3] {\setstretch{1.0}\footnotesize{Standard errors are clustered on the state level.}}
\end{tablenotes}
\end{threeparttable}
\end{table}


\newpage

\begin{table}[H]
\centering\centering
\caption{Effect of Having Hispanic Last Name: Hispanics with non-Mexican Ancestry \label{tab:lastname-ed-reg-nonmex}}
\centering
\begin{threeparttable}
\begin{tabular}[t]{lcccc}
\toprule
  & \specialcell{(1) \\ Years of \\ Education} & \specialcell{(2) \\ High School \\ Dropout} & \specialcell{(3) \\ Associate Degree} & \specialcell{(4) \\ Bachelor Degree}\\
\midrule
\addlinespace[0.5em]
\multicolumn{5}{l}{\textit{Panel A: Full Sample}}\\
\midrule \hspace{1em}$HW_{ist}$ & -0.25*** & 0.01 & -0.04*** & -0.04***\\
\hspace{1em} & (0.07) & (0.01) & (0.01) & (0.01)\\
\hspace{1em}Observations & 37893 & 38372 & 24567 & 38372\\
\addlinespace[0.5em]
\multicolumn{5}{l}{\textit{Panel B: Women}}\\
\midrule \hspace{1em}$HW_{ist}$ & -0.35*** & 0.03** & -0.04** & -0.06***\\
\hspace{1em} & (0.11) & (0.02) & (0.02) & (0.02)\\
\hspace{1em}Observations & 19774 & 20039 & 12476 & 20039\\
\addlinespace[0.5em]
\multicolumn{5}{l}{\textit{Panel C: Men}}\\
\midrule \hspace{1em}$HW_{ist}$ & -0.24** & 0.00 & -0.04* & -0.04*\\
\hspace{1em} & (0.10) & (0.02) & (0.02) & (0.02)\\
\hspace{1em}Full Sample's Mean & 14.07 & 0.38 & 0.16 & 0.36\\
\hspace{1em}Observations & 18119 & 18333 & 12091 & 18333\\
Women's Mean & 14.17 & 0.38 & 0.17 & 0.38\\
Men's Mean & 13.97 & 0.38 & 0.16 & 0.34\\
\bottomrule
\multicolumn{5}{l}{\rule{0pt}{1em}* p $<$ 0.1, ** p $<$ 0.05, *** p $<$ 0.01}\\
\end{tabular}
\begin{tablenotes}
\item[1] {\setstretch{1.0}\footnotesize{This table includes the estimation results of equation (\ref{eq:1a}). All regressions include state-year fixed effects.}}
\item[2] {\setstretch{1.0}\footnotesize{HW is an indicator variable that is equal to 1 if a person is the child of a Hispanic-father and White-mother.}}
\item[3] {\setstretch{1.0}\footnotesize{Standard errors are clustered on the state level.}}
\end{tablenotes}
\end{threeparttable}
\end{table}


\newpage

\begin{table}[H]
\centering\centering
\caption{Effect of Having Hispanic Last Name: Hispanics with Mexican Ancestry  \label{tab:lastnamereg-weekearm-mex}}
\centering
\begin{threeparttable}
\begin{tabular}[t]{lcccc}
\toprule
  & \specialcell{(1) \\ Log weekly \\ earnings} & \specialcell{(2) \\ Log weekly \\ earnings} & \specialcell{(3) \\  Log weekly \\ earnings} & \specialcell{(4) \\  Log weekly \\ earnings}\\
\midrule
$HW_{ist}$ & \num{-0.01} & \num{-0.04}** & \num{-0.01} & \num{-0.01}\\
 & (\num{0.02}) & (\num{0.02}) & (\num{0.01}) & (\num{0.02})\\
Constant & \num{6.53}*** &  &  & \\
 & (\num{0.04}) &  &  & \\
\midrule
\textit{Controlling for:} &  &  &  & \\
State FE &  & X & X & X\\
Year FE &  & X & X & X\\
State-Year FE &  & X & X & X\\
Age &  & X & X & X\\
Education &  &  & X & X\\
Parental Background &  &  &  & X\\
Observations & \num{4515} & \num{4515} & \num{4515} & \num{4515}\\
\bottomrule
\multicolumn{5}{l}{\rule{0pt}{1em}* p $<$ 0.1, ** p $<$ 0.05, *** p $<$ 0.01}\\
\end{tabular}
\begin{tablenotes}
\item[1] {\setstretch{1.0}\footnotesize{This table includes the estimation results of equation (\ref{eq:1a}) where the dependent variable is log weekly earnings.}}
\item[2] {\setstretch{1.0}\footnotesize{HW is an indicator variable that is equal to 1 if a person is the child of a Hispanic-father and White-mother.}}
\item[3] {\setstretch{1.0}\footnotesize{The sample is restricted to men working full-time and are wage and salary workers.}}
\item[4] {\setstretch{1.0}\footnotesize{Column one has the regression results when controlling for hours worked, age, education, year and state fixed effects. Column two has the results after controlling for education.}}
\item[5] {\setstretch{1.0}\footnotesize{Standard errors are clustered on the state level.}}
\end{tablenotes}
\end{threeparttable}
\end{table}


\newpage

\begin{table}[H]
\centering\centering
\caption{Effect of Having Hispanic Last Name: Hispanics with Non-Mexican Ancestry  \label{tab:lastnamereg-weekearm-nonmex}}
\centering
\begin{threeparttable}
\begin{tabular}[t]{lccc}
\toprule
  & \specialcell{(1) \\ Log weekly \\ earnings} & \specialcell{(2) \\ Log weekly \\ earnings} & \specialcell{(3) \\  Log weekly \\ earnings}\\
\midrule
$HW_{ist}$ & \num{-0.05}* & \num{-0.07}*** & \num{-0.04}**\\
 & (\num{0.03}) & (\num{0.02}) & (\num{0.02})\\
Constant & \num{6.75}*** &  & \\
 & (\num{0.03}) &  & \\
\midrule
\textit{Controlling for:} &  &  & \\
State FE &  & X & X\\
Year FE &  & X & X\\
State-Year FE &  & X & X\\
Age &  & X & X\\
Education &  &  & X\\
Parental Background &  &  & \\
Observations & \num{3225} & \num{3225} & \num{3225}\\
\bottomrule
\multicolumn{4}{l}{\rule{0pt}{1em}* p $<$ 0.1, ** p $<$ 0.05, *** p $<$ 0.01}\\
\end{tabular}
\begin{tablenotes}
\item[1] {\setstretch{1.0}\footnotesize{This table includes the estimation results of equation (\ref{eq:1a}) where the dependent variable is log weekly earnings.}}
\item[2] {\setstretch{1.0}\footnotesize{HW is an indicator variable that is equal to 1 if a person is the child of a Hispanic-father and White-mother.}}
\item[3] {\setstretch{1.0}\footnotesize{The sample is restricted to men working full-time and are wage and salary workers.}}
\item[4] {\setstretch{1.0}\footnotesize{Column one has the regression results when controlling for hours worked, age, education, year and state fixed effects. Column two has the results after controlling for education.}}
\item[5] {\setstretch{1.0}\footnotesize{Standard errors are clustered on the state level.}}
\end{tablenotes}
\end{threeparttable}
\end{table}


\newpage

% \begin{table}[H]

\caption{Effect of Having Hispanic Last Name: Hispanics with Cuban Ancestry  \label{tab:lastnamereg-weekearm-cub}}
\centering
\begin{threeparttable}
\begin{tabular}[t]{lcccc}
\toprule
  & \specialcell{(1) \\ Log weekly \\ earnings} & \specialcell{(2) \\ Log weekly \\ earnings} & \specialcell{(3) \\  Log weekly \\ earnings} & \specialcell{(4) \\  Log weekly \\ earnings}\\
\midrule
$HW_{ist}$ & \num{-0.08}** & \num{-0.03} & \num{-0.01} & \num{-0.02}\\
 & (\num{0.04}) & (\num{0.04}) & (\num{0.04}) & (\num{0.04})\\
Constant & \num{6.69}*** &  &  & \\
 & (\num{0.05}) &  &  & \\
\midrule
\textit{Controlling for:} &  &  &  & \\
State FE &  & X & X & X\\
Year FE &  & X & X & X\\
Age FE &  & X & X & X\\
Education FE &  &  & X & X\\
Parental Background &  &  &  & X\\
Observations & \num{1182} & \num{1182} & \num{1182} & \num{1182}\\
\bottomrule
\multicolumn{5}{l}{\rule{0pt}{1em}* p $<$ 0.1, ** p $<$ 0.05, *** p $<$ 0.01}\\
\end{tabular}
\begin{tablenotes}
\item[1] {\setstretch{1.0}\footnotesize{This table includes the estimation results of equation (\ref{eq:1a}) where the dependent variable is log weekly earnings.}}
\item[2] {\setstretch{1.0}\footnotesize{HW is an indicator variable that is equal to 1 if a person is the child of a Hispanic-father and White-mother.}}
\item[3] {\setstretch{1.0}\footnotesize{The sample is restricted to men working full-time and are wage and salary workers.}}
\item[4] {\setstretch{1.0}\footnotesize{Column one has the regression results when controlling for hours worked, age, education, year and state fixed effects. Column two has the results after controlling for education.}}
\item[5] {\setstretch{1.0}\footnotesize{Standard errors are clustered on the state level.}}
\end{tablenotes}
\end{threeparttable}
\end{table}


\begin{table}[H]
\centering\centering
\caption{Descriptive Statistics About Selection Into Occupation by Type \label{tab:occ-sum}}
\centering
\resizebox{\ifdim\width>\linewidth\linewidth\else\width\fi}{!}{
\begin{threeparttable}
\begin{tabular}[t]{>{}lcccc}
\toprule
\multicolumn{1}{c}{ } & \multicolumn{1}{c}{Hispanic Last Name} & \multicolumn{1}{c}{White Last Name} & \multicolumn{2}{c}{Difference-in-means: HW - WH} \\
\cmidrule(l{3pt}r{3pt}){2-2} \cmidrule(l{3pt}r{3pt}){3-3} \cmidrule(l{3pt}r{3pt}){4-5}
Occupation & Hispanic-White & White-Hispanic & Difference & p value\\
\midrule
\textbf{Management and Business} & 0.111 & 0.129 & -0.018 & 0.000\\
\textbf{STEM Occupations} & 0.030 & 0.040 & -0.009 & 0.000\\
\textbf{Healthcare Occupations} & 0.063 & 0.058 & 0.005 & 0.002\\
\textbf{Education and Social Services} & 0.062 & 0.064 & -0.002 & 0.237\\
\textbf{Arts, Media, and Entertainment} & 0.028 & 0.034 & -0.006 & 0.000\\
\addlinespace
\textbf{Service Occupations} & 0.107 & 0.099 & 0.008 & 0.000\\
\textbf{Manual and Industrial Labor} & 0.199 & 0.171 & 0.028 & 0.000\\
\bottomrule
\end{tabular}
\begin{tablenotes}
\item[1] Source: Current Population Survey (CPS) 1994-2019.
\end{tablenotes}
\end{threeparttable}}
\end{table}


\newpage

\begin{table}[H]
\centering\centering
\caption{Effect of Having Hispanic Last Name (Log Annual Earnings) \label{tab:lastnamereg-occ}}
\centering
\begin{threeparttable}
\begin{tabular}[t]{lccccc}
\toprule
  & \specialcell{(1) \\ Log annual \\ earnings} & \specialcell{(2) \\ Log annual \\ earnings} & \specialcell{(3) \\ Log annual \\ earnings} & \specialcell{(4) \\  Log annual \\ earnings} & \specialcell{(5) \\  Log annual \\ earnings}\\
\midrule
$HW_{ist}$ & \num{-0.05}*** & \num{-0.05}** & \num{-0.05}* & \num{-0.02} & \num{-0.01}\\
 & (\num{0.02}) & (\num{0.02}) & (\num{0.02}) & (\num{0.02}) & (\num{0.02})\\
Constant & \num{10.42}*** & \num{9.46}*** &  &  & \\
 & (\num{0.04}) & (\num{0.07}) &  &  & \\
\midrule
\textit{Controlling for:} &  &  &  &  & \\
Hours Worked &  & X & X & X & X\\
Age &  &  & X & X & X\\
State FE &  &  & X & X & X\\
Year FE &  &  & X & X & X\\
State-Year FE &  &  & X & X & X\\
Occupation FE &  &  &  & X & X\\
Parental Background &  &  &  &  & X\\
Observations & \num{3621} & \num{3621} & \num{3621} & \num{3487} & \num{3487}\\
\bottomrule
\multicolumn{6}{l}{\rule{0pt}{1em}* p $<$ 0.1, ** p $<$ 0.05, *** p $<$ 0.01}\\
\end{tabular}
\begin{tablenotes}
\item[1] {\setstretch{1.0}\footnotesize{This table includes the estimation results of equation (\ref{eq:1a}).}}
\item[2] {\setstretch{1.0}\footnotesize{HW is an indicator variable that is equal to 1 if a person is the child of a Hispanic-father and White-mother.}}
\item[3] {\setstretch{1.0}\footnotesize{The sample is restricted to men working full-time full-year and are wage and salary workers.}}
\item[4] {\setstretch{1.0}\footnotesize{Column one has the regression results when controlling for hours worked, age, education, year and state fixed effects. Column two has the results after controlling for education.}}
\item[5] {\setstretch{1.0}\footnotesize{Standard errors are clustered on the state level.}}
\end{tablenotes}
\end{threeparttable}
\end{table}


\restoregeometry


\newpage
\pagebreak

% Appendix Bibliography
\begingroup
\setstretch{1.0}
\setlength\bibitemsep{0pt}
\printbibliography[title=References for Online Appendix]
\endgroup
\pagebreak
\end{refsection}

\end{document}